\documentclass[12pt]{beamer}
\usetheme{moloch}

% ========== Packages ==========
\usepackage{amsmath}
\usepackage{mathtools}
\usepackage{amssymb}
\usepackage{amsthm}
\usepackage{tikz}
\usepackage{pgfplots}

% ========== TikZ Configuration ==========
\pgfplotsset{compat=1.18}
\usetikzlibrary{
    arrows, 
    calligraphy, 
    decorations.pathreplacing,
    patterns,
    calc,
    arrows.meta,
    positioning
}

% ========== Plot Marks for Dots ==========
\pgfplotsset{
    holdot/.style={color=red, fill=white, only marks, mark=*},
    soldot/.style={color=red, fill=red, only marks, mark=*}
}

% ========== Beamer Color Customization ==========
\setbeamercolor{progress bar}{fg=purple!60!white, bg=black!20!white}
\setbeamercolor{section in toc}{fg=black}
\setbeamercolor{subsection in toc}{fg=black!70}
\setbeamercolor{section title}{fg=white}
\setbeamercolor{alerted text}{fg=purple!60!white}

% ========== Progress Bar Line Widths ==========
\makeatletter
\setlength{\moloch@titleseparator@linewidth}{1pt}
\setlength{\moloch@progressonsectionpage@linewidth}{1pt}
\setlength{\moloch@progressinheadfoot@linewidth}{1pt}
\makeatother

% ========== Section Slides ==========
\AtBeginSection[]{
    \begin{frame}[standout]
        \sectionpage
    \end{frame}
}

% ========== Page Layout ==========
\setbeamersize{text margin left=0.5cm, text margin right=0.5cm}
\setbeamertemplate{footline}{}

% ========== Table of Contents as List ==========
\setbeamertemplate{section in toc}{%
    \leavevmode\leftskip=2em%
    \llap{\raisebox{0.2ex}{\textbullet}\kern1em}%
    \inserttocsection\par
}

% ========== Document Metadata ==========
\title{The Formal Definition of the Limit}
\subtitle{}
\author{Maths Society}
\date{2025-11-13}
\institute{}

% ========== Document Begin ==========
\begin{document}

\maketitle

\begin{frame}{Overview}
    \tableofcontents[hideallsubsections]
\end{frame}

\section{Motivation}

\begin{frame}{Creating an Informal Definition}
    Limits capture the idea of closeness.

    \vspace{0.3cm}

    The limit of a function tells you what number that function `approaches' at a particular point.

    \vspace{0.6cm}

    \textbf{Informal Definition:} $\lim_{x \to a} f(x) = L$ if, as $x$ approaches $a$, $f(x)$ approaches $L$.

    \vspace{0.6cm}

    To put this in other words, we can make $f(x)$ as close to $L$ as we want by making $x$ close enough to $a$.
\end{frame}

\begin{frame}{When the Informal Definition Fails}
    \textbf{Pathological Function:} $f(x) = \sin\left(\dfrac{1}{x}\right)$

    \vspace{0.3cm}
    \begin{center}
        \begin{tikzpicture}
            \begin{axis}[
                axis lines=center,
                xlabel={$x$},
                ylabel={$y$},
                xmin=-6.5, xmax=6.5,
                ymin=-1.5, ymax=1.5,
                ytick={-1,0,1},
                width=11.5cm,
                height=5.5cm,
                xlabel style={at={(axis description cs:1.0,0.48)}, anchor=west},
                ylabel style={at={(axis description cs:0.48,1.0)}, anchor=south},
            ]
            
            % Function plot
            \addplot[blue!70!black, domain=-1/6:-1/0.02, samples=300, smooth] 
                ({1/(\x)}, {sin(\x r)});
            \addplot[blue!70!black, domain=1/0.02:1/6, samples=300, smooth] 
                ({1/(\x)}, {sin(\x r)});
            
            \end{axis}
        \end{tikzpicture}
    \end{center}
\end{frame}

\begin{frame}{Ambiguities \& Questions}
    \begin{itemize}
        \item \textbf{``Arbitrarily close'':} How close exactly? Does it matter?
        \vspace{0.3cm}
        \item \textbf{``Gets close'':} Do we just need one value of $x$ that makes it close, or many?
        \vspace{0.3cm}
        \item \textbf{Oscillatory functions:} $\sin(1/x)$ gets close to every value in $[-1,1]$ as $x \to 0$.
        \vspace{0.3cm}
        \item \textbf{Proof requirements:} We need proper logical statements in order to write proper proofs.
    \end{itemize}
\end{frame}

\section{Building the Definition}

\begin{frame}
    We need to turn our intuition about limits into a proper, formal definition.

    \vspace{0.3cm}

    It makes sense that, the closer $x$ gets to $a$, the closer $f(x)$ gets to $L$.

    \vspace{0.6cm}

    \uncover<2>{
        \textbf{First step forwards:}

        \begin{center}
            \large
            $x$ is `close enough' to $a \implies f(x)$ is `close enough' to $L$
        \end{center}

        \vspace{0.6cm}

        If we can define `close enough', we are done.
    }
\end{frame}

\begin{frame}{Defining Closeness}
    The difference between two numbers tells you how close they are.

    \vspace{0.3cm}

    \begin{center}
        \begin{tikzpicture}
            \draw[very thick, {latex}-{latex}] (0,0) -- (6,0);
            \foreach \x/\l in {1,...,5}
                \draw (\x,3pt) -- (\x,-3pt) node[below]{$\l$};
            \draw[decorate, decoration={brace, amplitude=5pt, mirror, raise=4ex}]
                (2,0) -- (5,0) node[midway, yshift=-3em]{$5-2$};
        \end{tikzpicture}
    \end{center}

    But both $a-b$ and $b-a$ tell us how close $a$ and $b$ are to each other. The sign does not matter.

    \vspace{0.3cm}

    We use the absolute value, $|a-b|$.

    \begin{equation*}
        |x| \coloneqq \begin{cases}
            \phantom{-}x & \text{if } x \geq 0 \\
            -x & \text{if } x < 0
        \end{cases}
    \end{equation*}
\end{frame}

\begin{frame}
    We have two distances we are concerned with, which we can now quantify:

    \begin{itemize}
        \item $|x-a|$
        \vspace{0.3cm}
        \item $|f(x)-L|$
    \end{itemize}
    
    \vspace{0.3cm}

    Suppose we have to make $f(x)$ be within 0.1 of $L$.
    This means we need $|f(x)-L| < 0.1$.

    \vspace{0.3cm}

    If $x$ being 0.01 within $a$ guaranteed this is true, then we would write:

    \begin{equation*}
        |x-a| < 0.01 \implies |f(x)-L| < 0.1
    \end{equation*}

    \vspace{0.3cm}

    That statement means: if the first inequality is true, the second equality \alert{must} be true.
\end{frame}

\begin{frame}{Introducing Symbols}
    The distances we have are pretty important, and we need a better way to talk about them than just making up numbers.

    \vspace{0.3cm}

    \begin{itemize}
        \item \textbf{$\varepsilon$ (epsilon):} Represents some distance from $L$
       
        \vspace{0.3cm}

        Think of $\varepsilon$ as the ``tolerance'' or ``error margin''

        \vspace{0.3cm}
        
        \item \textbf{$\delta$ (delta):} Represents some distance from $a$
        
        \vspace{0.3cm}

        We choose this to be as small as we need
    \end{itemize}

    \vspace{0.5cm}

    Now we have $|x-a| < \delta$ and $|f(x)-L| < \varepsilon$.
    These are not new ideas, just labels for pre-existing ones.
\end{frame}

% Define custom function for visualization
\pgfmathdeclarefunction{myfunct}{1}{%
    \pgfmathparse{2*tan(deg(#1))+1}%
}

\begin{frame}{Visualizing $\boldsymbol{\varepsilon}$-$\boldsymbol{\delta}$}
    \begin{center}
        \begin{tikzpicture}[>=stealth]
            \begin{axis}[
                black,
                axis x line=middle,
                axis y line=center,
                every axis x label/.style={
                    at={(current axis.right of origin)}, 
                    anchor=north
                },
                every axis y label/.style={
                    at={(current axis.above origin)}, 
                    anchor=east
                },
                xmin=-1, xmax=1.3,
                ymin=-1, ymax=5,
                xtick=\empty,
                ytick=\empty,
                xlabel={$x$},
                ylabel={$y$},
                width=10cm,
            ]

            % Draw the plot
            \addplot[red, samples=100, domain=-1:1.3] {myfunct(x)};

            % Define coordinates
            \def\xb{0.75}
            \pgfmathsetmacro{\yb}{myfunct(\xb)}
            \path (axis cs:\xb, \yb) coordinate (1);
            
            % Define epsilon
            \pgfmathsetmacro{\eps}{0.8}
            \pgfmathsetmacro{\yLower}{\yb - \eps}
            \pgfmathsetmacro{\yUpper}{\yb + \eps}
            
            % Solve for x-values where f(x) = L ± epsilon
            \pgfmathsetmacro{\xa}{rad(atan((\yLower - 1) / 2))}
            \pgfmathsetmacro{\xc}{rad(atan((\yUpper - 1) / 2))}
            
            % Define points on curve
            \pgfmathsetmacro{\yaOnCurve}{myfunct(\xa)}
            \pgfmathsetmacro{\ycOnCurve}{myfunct(\xc)}
            
            \path (axis cs:\xa, \yaOnCurve) coordinate (0);
            \path (axis cs:\xc, \ycOnCurve) coordinate (2);
            \path (axis cs:0, 0) coordinate (origin);
            \end{axis}

            % Draw vertical and horizontal lines
            \tikzset{marker/.style={shorten <=-3pt, shorten >=-3pt}}
            \draw[marker, dashed, blue] (origin-|0) -- (0);
            \draw[marker, dashed, blue] (origin-|1) -- (1);
            \draw[marker, dashed, blue] (origin-|2) -- (2);
            \draw[marker, dashed, blue] (origin|-0) -- (0);
            \draw[marker, dashed, blue] (origin|-1) -- (1);
            \draw[marker, dashed, blue] (origin|-2) -- (2);

            % δ, ε labels
            \path (origin) ++(10pt,10pt) coordinate (offset);

            % Epsilon: symmetric distances on y-axis
            \draw[<->, black!50!blue] (offset|-0) -- node[right] {$\varepsilon$} (offset|-1);
            \draw[<->, black!50!blue] (offset|-1) -- node[right] {$\varepsilon$} (offset|-2);
            \node at (origin|-1) [left, xshift=-3pt, black!50!blue] {$L$};

            % Delta: asymmetric distances on x-axis
            \draw[<->, red!50!blue] (offset-|0) -- node[above] {$\delta_1$} (offset-|1);
            \draw[<->, red!50!blue] (offset-|1) -- node[above] {$\delta_2$} (offset-|2);
            \node at (origin-|1) [below, yshift=-3pt, red!50!blue] {$a$};
        \end{tikzpicture}
    \end{center}
\end{frame}

\begin{frame}
    From the previous graph, we can see that if $x$ is within the $\delta$-band around $a$, $f(x)$ will be in the $\varepsilon$-band around $L$.

    \vspace{0.3cm}

    Formally, $|x-a| < \delta \implies |f(x)-L| < \varepsilon$.

    \vspace{0.3cm}

    But our graph had two different distances from $a$, $\delta_1$ and $\delta_2$.
    We pick the smaller one to be $\delta$.
    This gives a stricter band around $a$, but that is fine.

    \vspace{0.3cm}

    \begin{equation*}
        \delta = \min(\delta_1, \delta_2)
    \end{equation*}

    \vspace{0.3cm}

    If it works ($|f(x)-L| < \varepsilon$) for one band, it will work for a less strict band.
\end{frame}

\begin{frame}
    The inequalities we had on the previous slide are not complete.
    They require $|f(x)-L| < \varepsilon$ to be true when $|x-a| = 0$, or $x=a$.

    \vspace{0.3cm}

    Remember that limits tell you about what a function approaches as $x$ approaches a certain number.
    It does not matter what happens when $x=a$.
    So, the first inequality is actually $0 < |x-a| < \delta$.

    \vspace{0.3cm}

    \begin{center}
        \begin{minipage}[c]{0.48\textwidth}
            \begin{tikzpicture}[>=stealth]
                \begin{axis}[
                    black,
                    axis x line=middle,
                    axis y line=center,
                    every axis x label/.style={
                        at={(current axis.right of origin)}, 
                        anchor=north
                    },
                    every axis y label/.style={
                        at={(current axis.above origin)}, 
                        anchor=east
                    },
                    xmin=-0.7, xmax=1.3,
                    ymin=-0.8, ymax=1.5,
                    xtick=\empty,
                    ytick=\empty,
                    xlabel={$x$},
                    ylabel={$y$},
                    width=7cm,
                ]

                % Draw the plot
                \addplot[red, samples=100, domain=-0.5:1.3] {x};
                
                % Add hole at x = a (on the line)
                \addplot[holdot] coordinates{(0.75,0.75)};
                
                % Add solid dot below x-axis
                \addplot[soldot] coordinates{(0.75,-0.5)};
                
                % Add label for f(a)
                \node at (axis cs:0.75,-0.5) [right, xshift=3pt, black!50!blue] {$f(a)$};

                % Define coordinates
                \def\xb{0.75}
                \pgfmathsetmacro{\yb}{\xb}
                \path (axis cs:\xb, \yb) coordinate (1);
                
                % Define epsilon
                \pgfmathsetmacro{\eps}{0.4}
                \pgfmathsetmacro{\yLower}{\yb - \eps}
                \pgfmathsetmacro{\yUpper}{\yb + \eps}
                
                % For f(x) = x, we have x = y directly
                \pgfmathsetmacro{\xa}{\yLower}
                \pgfmathsetmacro{\xc}{\yUpper}
                
                % Points on curve
                \pgfmathsetmacro{\yaOnCurve}{\xa}
                \pgfmathsetmacro{\ycOnCurve}{\xc}
                
                \path (axis cs:\xa, \yaOnCurve) coordinate (0);
                \path (axis cs:\xc, \ycOnCurve) coordinate (2);
                \path (axis cs:0, 0) coordinate (origin);
                \end{axis}

                % Draw vertical and horizontal lines
                \tikzset{marker/.style={shorten <=-3pt, shorten >=-3pt}}
                \draw[marker, dashed, blue] (origin-|0) -- (0);
                \draw[marker, dashed, blue] (origin-|1) -- (1);
                \draw[marker, dashed, blue] (origin-|2) -- (2);
                \draw[marker, dashed, blue] (origin|-0) -- (0);
                \draw[marker, dashed, blue] (origin|-1) -- (1);
                \draw[marker, dashed, blue] (origin|-2) -- (2);

                % δ, ε labels
                \path (origin) ++(10pt,10pt) coordinate (offset);

                % Epsilon: symmetric distances on y-axis
                \draw[<->, black!50!blue] (offset|-0) -- node[right] {$\varepsilon$} (offset|-1);
                \draw[<->, black!50!blue] (offset|-1) -- node[right] {$\varepsilon$} (offset|-2);
                \node at (origin|-1) [left, xshift=-3pt, black!50!blue] {$L$};

                % Delta: asymmetric distances on x-axis
                \draw[<->, red!50!blue] (offset-|0) -- node[above] {$\delta_1$} (offset-|1);
                \draw[<->, red!50!blue] (offset-|1) -- node[above] {$\delta_2$} (offset-|2);
                \node at (origin-|1) [below, yshift=-3pt, red!50!blue] {$a$};
            \end{tikzpicture}
        \end{minipage}\hspace{0.6cm}\begin{minipage}[c]{0.38\textwidth}
            \[
                f(x) = \begin{cases}
                    x & \text{if } x \neq a \\
                    -0.5 & \text{if } x = a
                \end{cases}
            \]
        \end{minipage}
    \end{center}
\end{frame}

\begin{frame}{Quantifiers}
    We have the main part of our definition, but we have not put any conditions on $\varepsilon$ and $\delta$ yet.
    
    \vspace{0.3cm}
    
    We want to get as close to $L$ as we can with $f(x)$, so we want to make $\varepsilon$ as small as possible.
    Thus, the statement should be true for \alert{all} possible values of $\varepsilon$.
    Distances can only be positive, so we consider all $\varepsilon > 0$.

    \vspace{0.3cm}

    Given some value of $\varepsilon$, we need one value of $\delta$ which makes the aforementioned implication true.
    Just one is enough, so our condition is that a suitable $\delta$ \alert{exists}.
    It is a distance, so $\delta > 0$.

    \vspace{0.3cm}

    Technically, distances are nonnegative, but we can show that either $\varepsilon \geq 0$ or $\delta \geq 0$ is either unnecessary or gives a nonsense definition.
\end{frame}

\begin{frame}{The Formal Definition}
    \begin{block}{Definition (Limit of a Function)}
        Let $f: D \to \mathbb{R}$ be some function.
        Let $a$ and $x$ be real numbers within the domain $D$.
        Then,
        $$\lim_{x \to a} f(x) = L$$
        if and only if:
        \vspace{0.2cm}

        \textit{For all $\varepsilon > 0$, there exists a $\delta > 0$ such that for all $x$ in $D$,}
        $$0 < |x - a| < \delta \implies |f(x) - L| < \varepsilon$$
    \end{block}

    \vspace{0.3cm}

    \textbf{In plain English:} No matter how small the value of $\varepsilon$, we can find a value of $\delta$ that guarantees $f(x)$ stays within $\varepsilon$ of $L$.
\end{frame}

\begin{frame}{Quantifier Notation}
    Our definition is complete, but we can introduce some symbols to make it easier to work with.
        
    \vspace{0.5cm}

    \begin{itemize}
        \item \textbf{$\forall$ (``for all''):} Universal quantifier
        
        \vspace{0.3cm}
        $\forall \varepsilon > 0$ means ``for every $\varepsilon$ greater than zero''
        
        \vspace{0.3cm}
        
        \item \textbf{$\exists$ (``there exists''):} Existential quantifier
        
        \vspace{0.3cm}
        $\exists \delta > 0$ means ``there exists a $\delta$ greater than zero''
    \end{itemize}

    \vspace{0.5cm}

    \textbf{Compact Definition (but equivalent):}
    $$\forall \varepsilon > 0 \;\; \exists \delta > 0 \;\; \forall x \in D: \quad 0 < |x - a| < \delta \implies |f(x) - L| < \varepsilon$$
\end{frame}

\begin{frame}{Why Quantifier Order Matters}
    The order of $\forall$ and $\exists$ \textbf{matters}.

    \vspace{0.6cm}

    \textbf{Correct:} $\forall \varepsilon > 0 \;\; \exists \delta > 0$ \quad (For every $\varepsilon$, find $\delta$)

    \begin{itemize}
        \item $\delta$ can depend on $\varepsilon$, and typically does
        \item Different $\varepsilon$ values may need different $\delta$ values
        \item This is what we need for limits
    \end{itemize}

    \vspace{0.6cm}

    \textbf{Wrong:} $\exists \delta > 0 \;\; \forall \varepsilon > 0$ \quad (Find one $\delta$ for all $\varepsilon$)

    \begin{itemize}
        \item Would require a single $\delta$ to work for \textit{every} $\varepsilon$
        \item Generally impossible for non-constant functions
        \item This is a much stronger condition
    \end{itemize}
\end{frame}

\begin{frame}{Logical Structure}
    \begin{center}
        \begin{tikzpicture}[
            node distance=1.1cm,
            every node/.style={align=center},
            boxstyle/.style={
                draw, 
                rounded corners, 
                minimum width=3.5cm, 
                minimum height=0.75cm, 
                font=\large
            },
        ]

        % Nodes
        \node[boxstyle, fill=blue!10] (forall) {$\forall \varepsilon > 0$};
        \node[boxstyle, fill=green!10, below=of forall] (exists) {$\exists \delta > 0$};
        \node[boxstyle, fill=orange!10, below=of exists] (forallx) {$\forall x \in D$};
        \node[boxstyle, fill=red!10, minimum width=5.5cm, below=of forallx] (implies) 
            {$0 < |x-a| < \delta \implies |f(x) - L| < \varepsilon$};

        % Arrows
        \draw[-{Latex}, thick] (forall) -- (exists);
        \draw[-{Latex}, thick] (exists) -- (forallx);
        \draw[-{Latex}, thick] (forallx) -- (implies);
        \end{tikzpicture}
    \end{center}
\end{frame}

\section{Applying the Definition}

\begin{frame}{Strategy for Proving Limits}
    \textbf{To prove $\lim_{x \to a} f(x) = L$:}

    \vspace{0.5cm}

    \begin{enumerate}
        \item \textbf{Let $\varepsilon > 0$ be arbitrary}
        
        Start by assuming we're given some unspecified positive $\varepsilon$
        
        \vspace{0.3cm}
        
        \item \textbf{Work backwards (scratch work)}
        
        Start from $|f(x) - L| < \varepsilon$ and manipulate to find $|x - a| < $ something, which will be in terms of $\varepsilon$
        
        \vspace{0.3cm}
        
        \item \textbf{Choose $\delta$}
        
        Define $\delta$ based on your scratch work.
        It can be whatever you want, as long as it works out in the end.
        
        \vspace{0.3cm}
        
        \item \textbf{Complete the proof}
        
        Show that $0 < |x - a| < \delta$ implies $|f(x) - L| < \varepsilon$
    \end{enumerate}
\end{frame}

\begin{frame}{Example 1: Linear Function}
    \textbf{Proposition:} $\displaystyle\lim_{x \to 2} (3x + 1) = 7$

    \vspace{0.4cm}
    
    \uncover<2>{
        \textbf{Proof.} Let $\varepsilon > 0$ be arbitrary. 

        \vspace{0.3cm}
        
        \textit{Scratch work.} We need:
        \begin{align*}
            |(3x + 1) - 7| < \varepsilon 
            &\iff |3x - 6| < \varepsilon \\
            &\iff 3|x - 2| < \varepsilon \\
            &\iff |x - 2| < \frac{\varepsilon}{3}
        \end{align*}

        \textit{Choice of $\delta$.} Define $\delta \coloneqq \dfrac{\varepsilon}{3}$.

        \vspace{0.3cm}
        
        \textit{Verification.} Suppose $0 < |x - 2| < \delta$. Then:
        \begin{equation*}
            |(3x+1) - 7| = 3|x-2| < 3 \cdot \frac{\varepsilon}{3} = \varepsilon
        \end{equation*}
    }
\end{frame}

\begin{frame}{Example 2: Quadratic Function}
    \textbf{Proposition:} $\displaystyle\lim_{x \to 3} x^2 = 9$

    \vspace{0.3cm}

    \uncover<2>{
        \textbf{Proof.} Let $\varepsilon > 0$ be arbitrary.

        \vspace{0.4cm}
        
        \textit{Scratch work.} We have $|x^2 - 9| = |x - 3| \cdot |x + 3|$, but the factor $|x + 3|$ varies with $x$.

        \vspace{0.3cm}

        We bound the variable factor.
        If $|x - 3| < 1$, then $|x + 3| < 7$.
        Thus, $|x^2 - 9| < 7|x - 3|$, so we need $|x - 3| < \varepsilon/7$.

        \vspace{0.3cm}
        
        \textit{Choice of $\delta$.} Let $\delta \coloneqq \min\left\{1, \dfrac{\varepsilon}{7}\right\}$.

        \vspace{0.3cm}
        
        \textit{Verification.} If $0 < |x-3| < \delta$, then $|x-3| < 1$ and $|x-3| < \varepsilon/7$, so 
        \[
            |x^2 - 9| < 7 \cdot \frac{\varepsilon}{7} = \varepsilon.
        \]
    }
\end{frame}

\begin{frame}{Key Observations}
    \textbf{From the examples:}

    \vspace{0.6cm}

    \begin{itemize}
        \item $\delta$ typically depends on $\varepsilon$ (e.g., $\delta = \varepsilon/3$)
        
        \vspace{0.4cm}
        
        \item You can have $\delta$ be the minimum of multiple different expressions if needed
        
        \vspace{0.4cm}
        
        \item Sometimes need to bound variable expressions (linked to using minimums)
        
        \vspace{0.4cm}
        
        \item The choice of $\delta$ isn't unique, and just one is good enough
        
        \vspace{0.4cm}
        
        \item Finding the right value of $\delta$ involves working backwards from the desired conclusion
    \end{itemize}
\end{frame}

\section{When Limits Fail to Exist}

\begin{frame}{Negating the Definition}
    To negate a logical statement, we `peel back' the layers.

    \vspace{0.3cm}

    The opposite of a `there exists \ldots\ is true' statement is a `for all \ldots\ is false' statement, and vice versa.
    We can go through and reverse each part.

    \vspace{0.5cm}

    \begin{block}{Non-Existence}
        $\displaystyle\lim_{x \to a} f(x) \neq L$ if and only if:

        \vspace{0.2cm}

        There exists $\varepsilon_0 > 0$ such that for all $\delta > 0$, there exists $x \in D$ with $0 < |x - a| < \delta$ and $|f(x) - L| \geq \varepsilon_0$
    \end{block}

    \vspace{0.3cm}

    \textbf{Compact Version:} 
    $\exists \varepsilon_0 > 0 \;\; \forall \delta > 0 \;\; \exists x \in D: \;\; 0 < |x - a| < \delta \;\wedge\; |f(x) - L| \geq \varepsilon_0$
\end{frame}

\begin{frame}{Strategy for Disproving Limits}
    \textbf{To show $\lim_{x \to a} f(x)$ does not exist (by contradiction):}

    \vspace{0.5cm}

    \begin{enumerate}
        \item \textbf{Assume the limit exists}
        
        Suppose $\lim_{x \to a} f(x) = L$ for some $L \in \mathbb{R}$
        
        \vspace{0.3cm}
        
        \item \textbf{Find a problematic $\varepsilon$}
        
        Choose $\varepsilon=\varepsilon_0$ such that $f$ behaves `badly' irrespective of $\delta$
        
        \vspace{0.3cm}
        
        \item \textbf{Show the definition fails}
        
        Demonstrate that no matter what $\delta$ is chosen, there exist points $x$ near $a$ where $|f(x) - L| \geq \varepsilon_0$
        
        \vspace{0.3cm}
        
        \item \textbf{Complete disproof:}
        
        State that there is a contradiction and thus the original assumption (of existence) is false
    \end{enumerate}
\end{frame}

\begin{frame}{Pathological Case I: Unbounded Oscillation}
    \textbf{Function:} $f(x) = \sin\left(\dfrac{1}{x}\right)$

    \vspace{0.3cm}

    What is the behaviour of the function as $x \to 0$?
    We know that $1/x \to \infty$, but $\sin x$ does not approach anything as $x \to \infty$.
    The graph of $f(x)$ shown before suggests it has no limit.

    \vspace{0.3cm}

    How can we prove that?

    \vspace{0.3cm}

    \uncover<2>{
        Suppose the limit exists and equals some real number $L$.
        This means that 
        \begin{equation*}
            \forall \varepsilon > 0 \;\; \exists \delta > 0 \;\; \forall x \in \mathbb{R}^* \colon \;\; 0 < |x| < \delta \implies |\sin(1/x) - L| < \varepsilon
        \end{equation*}

        \textbf{Note:} $\mathbb{R}^* = \mathbb{R} \setminus \{0\}$ is the domain of $f$.
    }
\end{frame}

\begin{frame}{Pathological Case I: Proof}
    Let $\varepsilon = 1/2$, and let $\delta > 0$ satisfy the previous implication.

    \vspace{0.3cm}
    
    Let $x_1 = \frac{1}{2\pi n + \frac{\pi}{2}}$ and $x_2 = \frac{1}{2\pi n + \frac{3\pi}{2}}$ with $n \in \mathbb{Z}^+$.
    It follows that $0 < x_2 < x_1$.
    We want $\frac{1}{2\pi n + \frac{\pi}{2}} < \delta$ which requires $n > \frac{1}{2\delta \pi} - \frac{1}{4}$.

    \vspace{0.3cm}

    Let $n$ satisfy that inequality, so $0 < |x_1| < \delta \implies |\sin(1/x_1) - L| < 1/2 \iff |1 - L| < 1/2$, and 

    \vspace{0.2cm}

    $0 < |x_2| < \delta \implies |\sin(1/x_2) - L| < 1/2 \iff |-1 - L| < 1/2 \iff |1 + L| < 1/2$.

    \vspace{0.3cm}

    Adding these inequalities gives $|1 - L| + |1 + L| < 1$.

    Also, $2 = |(1-L) + (1+L)| \leq |1-L| + |1+L|$ by the triangle inequality.
    This implies that $2 < 1$, which is a contradiction.

    \vspace{0.3cm}

    Therefore, the limit does not exist.
    Since $L$ was arbitrary, the function has no limit as $x \to 0$.
\end{frame}

\begin{frame}{Pathological Case II: Jump Discontinuity}
    \textbf{Function (Heaviside Step):} 
    $H(x) = \begin{cases} 0 & \text{if } x < 0 \\ 1 & \text{if } x \geq 0 \end{cases}$ 
    at $x = 0$

    \vspace{0.2cm}

    \begin{center}
        \begin{minipage}[c]{0.50\textwidth}
            \small
            \uncover<2>{
                Assume the limit as $x \to 0$ is $L \in \mathbb{R}$.
                Let $\varepsilon = 1/2$, and follow the procedure of the previous example.

                \vspace{0.3cm}

                Let $x_1 = \delta/2$ and $x_2 = -\delta/2$.
                Summing and using the triangle inequality gives you $1 < 1$ as your contradiction.

                \vspace{0.3cm}

                Since $L$ is arbitrary, the limit doesn't exist.
            }
        \end{minipage}\hspace{0.4cm}\begin{minipage}[c]{0.46\textwidth}
            \begin{tikzpicture}[>=stealth]
                \begin{axis}[
                    black,
                    axis x line=middle,
                    axis y line=center,
                    every axis x label/.style={
                        at={(current axis.right of origin)}, 
                        anchor=north
                    },
                    every axis y label/.style={
                        at={(current axis.above origin)}, 
                        anchor=east
                    },
                    xmin=-1.5, xmax=1.5,
                    ymin=-0.3, ymax=1.3,
                    ytick={1},
                    yticklabels={$1$},
                    xtick={-1,1},
                    xticklabels={$-1$,$1$},
                    xlabel={$x$},
                    ylabel={$y$},
                    width=6.5cm,
                ]
                
                % Piecewise function
                \addplot[red, line width=1.4pt, domain=-1.5:-0.01] {0};
                \addplot[red, line width=1.4pt, domain=0:1.5] {1};
                
                % Points at discontinuity
                \addplot[soldot] coordinates{(0,1)};
                \addplot[holdot] coordinates{(0,0)};
                \end{axis}
            \end{tikzpicture}
        \end{minipage}
    \end{center}
\end{frame}

\section{Extensions and Exercises}

\begin{frame}{One-Sided Limits}
    It makes sense to talk about what a function approaches from just the left or right sides, and we can formalize this as well.
    They might approach the same number, or they might not.

    \vspace{0.3cm}

    \begin{block}{Right-Hand Limit}
        $\displaystyle\lim_{x \to a^+} f(x) = L$ 
        
        \vspace{0.1cm}
        
        means: For every $\varepsilon > 0$, there exists $\delta > 0$ such that
        $0 < x - a < \delta \implies |f(x) - L| < \varepsilon$
    \end{block}

    \vspace{0.2cm}

    \begin{block}{Left-Hand Limit}
        $\displaystyle\lim_{x \to a^-} f(x) = L$ 
        
        \vspace{0.1cm}
        
        means: For every $\varepsilon > 0$, there exists $\delta > 0$ such that
        $0 < a - x < \delta \implies |f(x) - L| < \varepsilon$
    \end{block}

    \vspace{0.3cm}

    The changes to the original definition are very minor.
\end{frame}

\begin{frame}{Relationship Between Limits}
    A limit will only exist if and only if both the one-sided limits exist and are equal.

    \begin{equation*}
        \lim_{x \to a} f(x) = L \iff \lim_{x \to a^+} f(x) = L = \lim_{x \to a^-} f(x)
    \end{equation*}

    \vspace{0.3cm}

    Intuitively, this makes sense.
    A function needs to approach the same number from the left and right if we are to say it definitively approaches anything at all.

    \vspace{0.3cm}

    \textbf{Application to Heaviside function:}

    \vspace{0.2cm}

    For $H(x)$:

    \vspace{0.2cm}

    \begin{itemize}
        \item $\lim_{x \to 0^-} H(x) = 0$ (approaching from left)
        
        \vspace{0.1cm}

        \item $\lim_{x \to 0^+} H(x) = 1$ (approaching from right)
        
        \vspace{0.1cm}

        \item Since $0 \neq 1$, the two-sided limit $\lim_{x \to 0} H(x)$ does not exist
    \end{itemize}
\end{frame}

\begin{frame}{Connection to Other Concepts}
    \begin{center}
        \begin{tikzpicture}[
            concept/.style={
                rectangle, 
                rounded corners, 
                draw, 
                thick, 
                fill=blue!20, 
                minimum width=2.8cm, 
                minimum height=1cm, 
                align=center, 
                font=\small
            },
            arrow/.style={{Latex}-, thick, blue!60!black},
        ]

        % Central node
        \node[concept, fill=red!30, minimum width=3.2cm, minimum height=1.2cm, font=\normalsize] (limit) at (0,0) {
            {$\boldsymbol{\varepsilon}$-$\boldsymbol{\delta}$}\\
            \textbf{Limit Definition}
        };

        % Top concepts
        \node[concept] (continuity) at (-3,2.6) {
            \textbf{Continuity}\\[0.1cm]
            Continuous if limit exists
        };

        \node[concept] (derivative) at (3,2.6) {
            \textbf{Derivatives}\\[0.1cm]
            Defined by a limit
        };

        % Bottom concepts
        \node[concept] (sequences) at (-0.2,-3.2) {
            \textbf{Sequences}\\[0.1cm]
            $\lim_{n \to \infty} a_n = L$\\
            (analogous definition)
        };

        \node[concept] (integrals) at (-4.1,-2.5) {
            \textbf{Integrals}\\[0.1cm]
            Limit of\\
            Riemann sums
        };

        \node[concept] (series) at (4.1,-2.3) {
            \textbf{Series}\\[0.1cm]
            Limit of\\
            partial sums
        };

        % Arrows from central concept
        \draw[arrow] (continuity.south) -- (limit.north west);
        \draw[arrow] (derivative.south) -- (limit.north east);
        \draw[{Latex}-{Latex}, thick, blue!60!black] (sequences.north) -- (limit.south);
        \draw[arrow] (integrals.north east) -- (limit.south west);
        \draw[arrow] (series.north west) -- (limit.south east);

        % Additional connections
        \draw[{Latex}-, dashed, thick, black] (derivative.west) -- (continuity.east);
        \draw[{Latex}-, dashed, thick, black] (series.west) -- (sequences.east);
        \end{tikzpicture}
    \end{center}
\end{frame}

\begin{frame}{Exercises I}
    \textbf{Prove using the $\varepsilon$-$\delta$ definition:}

    \begin{enumerate}
        \item $\displaystyle\lim_{x \to 1} (x^2 + 2x - 1) = 2$
        \vspace{0.2cm}

        \item $\displaystyle\lim_{x \to 9} (\sqrt{x}) = 3$
        \vspace{0.2cm}
        
        \item $\displaystyle\lim_{x \to 2} \frac{x^2 - 4}{x - 2} = 4$
        \vspace{0.2cm}
        
        \item $\displaystyle\lim_{x \to 0} x^2 \sin\left(\frac{1}{x}\right) = 0$
    \end{enumerate}

    \vspace{0.4cm}

    \textbf{Prove:} $\displaystyle\lim_{x \to 0} \frac{1}{x}$ does not exist

    \vspace{0.4cm}

    \textbf{Prove the sum rule:} If $\lim_{x \to a} f(x) = L$ and $\lim_{x \to a} g(x) = M$, then $\lim_{x \to a} [f(x) + g(x)] = L + M$.
\end{frame}

\begin{frame}{Exercises II}
    \textbf{Definition (Continuity):} A function $f$ is continuous at a point $x=a$ if $\lim_{x \to a} f(x) = f(a)$.
    
    \vspace{0.3cm}

    \textbf{Prove:}

    \begin{enumerate}
        \item $f(x) = c$ is continuous for all $x \in \mathbb{R}$ ($c \in \mathbb{R}$)
        
        \vspace{0.1cm}

        \item $f(x) = x^2$ is continuous for all $x \in \mathbb{R}$
        
        \vspace{0.1cm}

        \item $f(x) = e^x$ is continuous for all $x \in \mathbb{R}$
    \end{enumerate}

    \vspace{0.3cm}

    \textbf{Prove:} $\displaystyle\lim_{x \to 0} \frac{\sin x}{x} = 1$

    \vspace{0.3cm}

    You are given that $\sin x \geq x - \frac{x^2}{2}$ for all $x \in [0,\pi]$.
\end{frame}

\begin{frame}{Summary}
    \begin{center}
        \begin{tikzpicture}
            \node[
                draw, 
                rounded corners, 
                fill=blue!15, 
                align=center, 
                minimum height=1.5cm, 
                inner sep=12pt
            ] (def) {
                \textbf{The $\varepsilon$-$\delta$ Definition of the Limit}\\[0.15cm]
                $\displaystyle\lim_{x \to a} f(x) = L$ means:\\[0.1cm]
                For all $\varepsilon > 0$, there exists a $\delta > 0$ such that for all $x$,\\[0.05cm]
                $0 < |x - a| < \delta \implies |f(x) - L| < \varepsilon$\\[0.05cm]
                OR\\[0.05cm]
                $\forall \varepsilon > 0 \;\; \exists \delta > 0 \;\; \forall x \colon \;\; 0 < |x - a| < \delta \implies |f(x) - L| < \varepsilon$
            };

            \node[below=0.3cm of def, text width=10.5cm] (principles) {
                \textbf{Key principles:}
                \begin{itemize}
                    \item Quantifier order is important: find $\delta$ for given $\varepsilon$
                    \item Work backwards: start from $|f(x) - L| < \varepsilon$
                    \item Pathological cases show why rigour is necessary
                    \item Foundation for all of calculus and real analysis
                \end{itemize}
            };
        \end{tikzpicture}
    \end{center}
\end{frame}

\end{document}