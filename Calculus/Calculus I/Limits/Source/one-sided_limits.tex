\documentclass[12pt]{article}

\usepackage{amsmath}
\usepackage{amssymb}
\usepackage{amsfonts}
\usepackage[style=iso]{datetime2}
\usepackage[explicit]{titlesec}
\usepackage{amsthm}
\usepackage{array}
\usepackage{graphicx}
\usepackage{float}

\graphicspath{ {./Images/} }

\theoremstyle{definition}
\newtheorem{problem}{Problem}
\newtheorem{definition}{Definition}

\begin{titlepage}
\title{Calculus I: One-Sided Limits}
\author{The Melon Man}
\date{\today}
\end{titlepage}

\renewcommand{\thesection}{\Roman{section}}

\allowdisplaybreaks

\setlength{\parindent}{0pt}
\setlength{\parskip}{1em}

\begin{document}
\maketitle

In the previous section, we looked at two limits which do not exist; the reason for them not existing, however, were quite different.
One of the limits were:

\begin{equation*}
    \lim_{t\to0} \cos\left(\frac{\pi}{t}\right)
\end{equation*}

We saw that this limit does not exist as the function oscillates wildly as we approach $t=0$ from either side.
As the function does not settle on any one value, the limit does not exist.
We had also considered the following limit.

\begin{equation*}
    \lim_{x \to 0} H(t), \quad H(t) =
    \begin{cases}
        0 & \text{if } t < 0    \\
        1 & \text{if } t \geq 0
    \end{cases}
\end{equation*}

This limit does not exist either but for a different reason.
From the left of $t=0$, $H(t)$ approaches 0 while it approaches 1 from the right.
As the function settles on two different values depending on which side of $t=0$ we are looking at, the limit does not exist.

We will differentiate these two cases of limits that do not exist.
This will be done with \textbf{one-sided limits} whose definitions are present below; these involve us looking at what a function approaches from only one side.

\begin{definition}
    The limit of $f(x)$ is $L$ as $x$ approaches $a$ from the right, written as

    \begin{equation*}
        \lim_{x\to a^+} f(x) = L, \label{eq:1}
    \end{equation*}

    if $f(x)$ can be made to be close to $L$ for all values of $x$ close to $a$, with $x>a$.
\end{definition}

\begin{definition}
    The limit of $f(x)$ is $L$ as $x$ approaches $a$ from the left, written as

    \begin{equation*}
        \lim_{x\to a^-} f(x) = L, \label{eq:2}
    \end{equation*}

    if $f(x)$ can be made to be close to $L$ for all values of $x$ close to $a$, with $x<a$.
\end{definition}

Notice that the change in notation here from normal limits is very slight.
The only difference is that there is a superscripted sign after the $a$, under the "lim" part; right-handed limits have us go to $a^+$ while left-handed limits have us go to $a^-$.
These tell us whether we are considering $x>a$ or $x<a$, the direction we approach the target value.

\end{document}