\documentclass[12pt]{article}

\usepackage{amsmath}
\usepackage{amssymb}
\usepackage{amsfonts}
\usepackage[style=iso]{datetime2}
\usepackage[explicit]{titlesec}

\begin{titlepage}
\title{Calculus I: Review of Functions}
\author{The Melon Man}
\date{\today}
\end{titlepage}

\renewcommand{\thesection}{\Roman{section}}

\counterwithout{subsection}{section}  % have subsections numbered independently of sections
\titleformat{\subsection}{\normalfont\large\bfseries}{}{0em}{#1\ \thesubsection}


\allowdisplaybreaks

\setlength{\parindent}{0pt}
\setlength{\parskip}{1em}

\begin{document}
\maketitle

\section{Introduction}
Functions are incredibly important in calculus.
They will appear in nearly every single section of this course and the later calculus courses.
Therefore, it is important to have a good understanding of what a function is and how to work with them.

What is a function?
A function maps elements from one set to another set.
Specifically, it maps exactly one element from the first set to exactly one element in the second set.
For instance, we may have the mapping $X \mapsto Y$ where $X$ and $Y$ are two sets.
Here, $X$ is the domain of the function and $Y$ is the codomain of the function.
The one-to-one mapping is important to the definition of the function.

\subsection{Example}
Here are two equations in terms of $x$ and $y$:

\begin{align}
    y = x^2 + 1 \\
    y^2 = x + 1 \label{eq:1}
\end{align}

The first equation is a function because for every value of $x$, there is exactly one value of $y$.
Thus, you would call $y$ a function of $x$.
However, the second equation is not a function of $x$ because the exponet is present on $y$.
This may be better illustrated by taking the square root of both sides of equation~\eqref{eq:1}:

\begin{align*}
    y = \pm \sqrt{x + 1}
\end{align*}

We can see that there are two values of $y$ for most values of $x$.
I say most because of instances such as $x = 0$ where there is only one value of $y$ (namely, $y = 1$).
All it takes is one instance of multiple values of $y$ for a given value of $x$ for the equation to not be a function of $x$.
Now, let's look at function notation.

\begin{align}
    \label{eq:2} y = 2x^2-5x+3
\end{align}

Instead of writing it as an equation equal to $y$, we may write some letter with our indepent variable in parentheses.
With this, equation~\eqref{eq:2} becomes $f(x) = 2x^2-5x+3$.
We can use any letter to denote a function, but $f$ is the most common.
If it is already used, then we may use $g$, $h$, or any other letter.

Remember that this does not mean some letter multiplied by $x$.
The parentheses are just a way to denote that the function is a function of $x$.

\section{Function Evaluation}
We may see the usefulness of this if we wished to find the value of our function at a specific value of $x$.
For instance, if we wanted to find the value of $f(x)$ when $x = 2$:

\begin{align*}
    f(2) & = 2(2)^2-5(2)+3 \\
         & = 2(4)-10+3     \\
         & = 8-10+3        \\
         & = 1
\end{align*}

where we were able to replace the $x$ in the parentheses with $2$, as well as all other instances of $x$ in the equation.
We can see how function notation let's us compactly represent functions at some value.
Let's evaluate another function at some points.


\subsection{Example}
Given that $f(x) = -x^2+6x-11$, find the following:

\begin{enumerate}
    \item $f(0)$
    \item $f(2)$
    \item $f(t)$
    \item $f(x-3)$
    \item $f(4x-1)$
\end{enumerate}

The answers are as follows:

\begin{enumerate}
    \item $-(0)^2+6(0)-11 = -11$
    \item $-(2)^2+6(2)-11 = 3$
    \item $-(t)^2+6(t)-11 = -t^2+6t-11$
    \item $-(x-3)^2+6(x-3)-11$ = $-x^2+12x-38$
    \item $-(4x-1)^2+6(4x-1)+11$ = $-16x^2+32x-18$
\end{enumerate}

\section{Roots}
Those were simple enough.
Quite frequently, we want to find the root of a function.
For a function $f(x)$, this is the value of $x$ which satisfies the equation $f(x)=0$.
Functions may have multiple values of $x$ that would satisfy such an equation, so those functions would have multiple roots.
Let's look at an example:

\subsection{Example}
Determine all of the roots of $f(t) = 9t^3 - 18t^2 + 6t$

We would solve this by first setting the function equal to 0:

\begin{equation}
    9t^3 - 18t^2 + 6t = 0
\end{equation}


We would then want to factor the equation as much as possible.
Here, we can factor $t$ out of the left hand side:

\begin{equation}
    t(9t^2 - 18t + 6) = 0
\end{equation}

If either $t$ or $9t^2 + 18t + 6$ is equal to $0$, the whole thing will be $0$.
Thus, we can solve them seperately to get our roots.
We can obviously get $t=0$ as a root.
For the quadratic:

\begin{align}
    9t^2 - 18t + 6 & = 0                                                \\
    \nonumber                                                           \\
    t              & = \frac{-b\pm\sqrt{b^2-4ac}}{2a}                   \\
    a              & = 9                                                \\
    b              & = -18                                              \\
    c              & = 6                                                \\
    \nonumber                                                           \\
    t              & = \frac{18 \pm \sqrt{18^2-4\cdot9\cdot6}}{2\cdot9} \\
    t              & = \frac{18 \pm \sqrt{108}}{18}                     \\
    t              & = 1 \pm \frac{6\sqrt{3}}{18}                       \\
    t              & = 1 \pm \frac{\sqrt{3}}{3}
\end{align}

The three roots of our equation are $0$, $1 + \frac{\sqrt{3}}{3}$ and $1 - \frac{\sqrt{3}}{3}$.
That was a rather simple example but we used quite a few important things, such as our factoring skills and the quadratic formula.

\section{Domains \& Ranges}
Another important thing to note about functions is the domain and range.
As stated before, the domain is any $x$ value we can put in the function that will give us a real value.
The range is any $x$ value that the function will output.

\subsection{Example}
Let's practice by finding the domains and ranges of the following functions:

\begin{enumerate}
    \item $f(x)$ = $5x-3$
    \item $g(t)$ = $\sqrt{4-7t}$
    \item $h(x)$ = $-2x^2+12x+5$
    \item $f(z)$ = $|z-6|-3$
    \item $g(x)$ = $8$
\end{enumerate}

For the first one, we notice that the function can take any real value of $x$ and will correspondingly output a real value.
Therefore, it has every real number in its domain. Formally, we would write that the domain of $f(x)$ is $(-\infty , \infty)$.
We also know that the function may output any number, so its range is $(-\infty , \infty)$ as well.

For the second function, $g(t)$, we know that the principal square root only outputs non-negative numbers.
Our range for $g(t)$ is thus $[0, \infty)$.
                The domain is a bit more difficult here.
                We must only only allow values of $t$ that satisfy the equation $4-7t\geq0$ so that the square root does not take in any negative numbers.
                We can rearrange this inequality:

                \begin{align}
                    4-7t        & \geq 0  \\
                    4           & \geq 7t \\
                    \frac{4}{7} & \geq t  \\
                    t \leq \frac{4}{7}
                \end{align}

                So, $t$ may be any number less than or equal to $\frac{4}{7}$.
                Our domain is thus $\left(-\infty, \frac{4}{7}\right]$.

The third function $h(x)$ is a quadratic polynomial.
We know that the domain is $(-\infty, \infty)$.
It will actually be the range that is more difficult to get.
We know that the parabola graphed from this quadratic polynomial will curve downwards as the coefficient of the $x^2$ term is negative.
We thus know that there will be a maximum $y$-value, which will be one of the limits of the range.
The vertex of the quadratic's parabola is given by $-\frac{b}{2a}$ which equals $3$ in this case.
Plugging that $x$ value into the original equation gives us a $y$-value of $23$, which is our maximum $y$-value.
Our range is then $(\infty, 23]$.

For the fourth function $f(z)$, the domain is $(-\infty , \infty)$ which is rather simple.
Once again, the range will take a bit more work.
We know that as $z$ becomes arbitrarily large, so will the value of $f(z)$.
However, the same will happen as $z$ becomes arbitrarily small due to the absolute value signs.
The smallest the value of the absolute value term is 0 due to its nature, which would be at $z=6$.
The value of $f(z)$ at that point is -3, so the range is $[-3, \infty)$.

The fifth function $g(x)$ is rather simple.
It can take in any value of $x$ and will return $8$ no matter what.
The domain is thus $(-\infty , \infty)$ and the range is $8$.

Generally, finding the range of a function can be quite difficult.
We have thus far restricted ourselves to rather simple functions while finding the ranges.
Finding the domains of functions is usually quite simpler.
Let's look at some functions with more complex domains, ignoring ranges for simplicity.

\subsection{Example}
Find the domains of the following functions:

\begin{enumerate}
    \item $f(x) = \displaystyle \frac{{x - 4}}{{{x^2} - 2x - 15}}$
    \item $g(t) = \sqrt {6 + t - {t^2}}$
    \item $h(x) = \displaystyle \frac{x}{{\sqrt {{x^2} - 9} }}$
\end{enumerate}

For $f(x)$, it may be helful to put it in a factored form:

$$
    f(x) = \frac{x - 4}{(x-5)(x+3)}
$$

At $x=5$ and $x=-3$, the denominator would be $0$ and we would have a division by $0$ problem.
The function will accept any other values of $x$ though.
Therefore, our domain for $f(x)$ is $(-\infty, \infty)$ for any real number $x$ except $-3$ and $5$.
This may also be written as $\mathbb{R} \setminus \left\{ -3, 5 \right\}$.


Let's do the second function $g(x)$.
We need to have a domain that alway satisfies the inequality $6 + t - t^2 \geq 0$.
We may solve for $t$ as follows:

\begin{align}
    6 + t - t^2 & \geq 0 \\
    t^2 - t - 6 & \leq 0 \\
    (t-3)(t+2)  & \leq 0
\end{align}

This means that the quadratic may change sign at $t=3$ and $t=-2$.
To find out, we will have to test the three regions created by them.
At $t=-3$, we get $6$, which does not satisfy the modified inequality.
The value at $t=0$ is $-6$ which does satisfy it.
Lastly, the value at $t=4$ is $6$ does not satisfy it.
We can thus see how the interval $[-2, 3]$ satisfies the modified inequality and will also satisfy the original inequality.
Therefore, our domain is $[-2, 3]$.

For the thrid function $h(x)$, we must make sure that $x^2-9$ is always positive for the function to exist.
This is to avoid the radicand being negative as well as a division by $0$ situation.
However, due to the exponent, we may realise that it will be at its lowest value at $x=0$.
The radicand would thus be positive for any value of $x$ that is greater than $3$ or smaller than $-3$ as either would be squared to get a number greater than $9$.
The domain is thus $(-\infty, -3)$ \& $(3, \infty)$.

\section{Function Composition}

This is plugging a function inside another. The compsition of the two functions $f(x)$ and $g(x)$ is:


\begin{equation}
    f(x) \circ g(x) = f(g(x))
\end{equation}

Note that the order matters as composition is not associative.
Generally, it would be true that some function $g(x)$ inside a function $f(x)$ would be different from $f(x)$ being inside $g(x)$.
Let's answer some problems with function composition involved.

\subsection{Example}
Given that $f(x) = 3x^2 - x + 10$ and $g(x) = 1 - 20x$, answer the following:

\begin{enumerate}
    \item $(f \circ g)(5)$
    \item $(f \circ g)(x)$
    \item $(g \circ f)(x)$
    \item $(g \circ g)(x)$
    \item $(f \circ f)(x)$
\end{enumerate}

Let's do the first problem. We can do it with $g(x)$ instead of $g(5)$ and plug in $x=5$ at the end.
We want to replace $x$ in $f(x)$ with $g(x)$, giving us:

\begin{align}
    (f \circ g)(x) & = 3(g(x))^2 - (g(x)) + 10           \\
                   & = 3(1 - 20x)^2 - (1 - 20x) + 10     \\
                   & = 3 - 120x + 1200x^2 - 1 + 20x + 10 \\
                   & = 1200x^2 - 100x + 12
\end{align}

At $x=5$:

\begin{equation}
    (f \circ g)(5) = 1200(5)^2 - 100(5) + 12 = 29512
\end{equation}

Note that we could also have evaluated $g(x)$ at $x=5$ and put that as the $x$ value into $f(x)$.
Either way, we get the same answer.
For the second problem, we already have our answer:

\begin{equation}
    (f \circ g)(x) = 1200x^2 - 100x + 12
\end{equation}

For the third question, we can use the same method used to get the answer to second one.

\begin{align}
    (g \circ f)(x) & = 1 - 20(f(x))          \\
                   & = 1 - 20(3x^2 - x + 10) \\
                   & = 1 - 60x^2 + 20x - 200 \\
                   & = -60x^2 + 20x - 199
\end{align}

Then our answer is:

\begin{equation}
    (g \circ f)(x) = -60x^2 + 20x - 199
\end{equation}

From that we can see that the order of the composition matters and can change the final result by quite a bit.
The fourth problem involves putting $g(x)$ inside itself.
We can do so as follows:

\begin{align}
    (g \circ g)(x) & = 1 - 20(g(x))    \\
                   & = 1 - 20(1 - 20x) \\
                   & = 1 - 20 + 400x   \\
                   & = 400x - 19
\end{align}

We know that this composition is:

\begin{equation}
    (g \circ g)(x) = 400x - 19
\end{equation}

For the last one, we do the same thing but with $f(x)$ instead.

\begin{align}
    (f \circ f)(x) & = 3(f(x))^2 - f(x) + 10                                   \\
                   & = 3(3x^2 - x + 10)^2 - (3x^2 - x + 10) + 10               \\
                   & = 27x^4 - 18x^3 + 180x^2 - 60x + 300 - 3x^2 + x - 10 + 10 \\
                   & = 27x^4 - 18x^3 + 180x^2 - 59x + 300
\end{align}

Our answer is:

\begin{equation}
    (f \circ f)(x) = 27x^4 - 18x^3 + 180x^2 - 59x + 300
\end{equation}

Let's do one more problem.
This will be a special case showing us a certain relation between two functions, which we will cover in a later section.

\subsection{Example}
Given that $f(x) = 3x - 2$ and $g(x) = \frac{1}{3}x + \frac{2}{3}$, find the following:

\begin{enumerate}
    \item $(f \circ g)(x)$
    \item $(g \circ f)(x)$
\end{enumerate}

The first question is solved as follows:

\begin{align}
    (f \circ g)(x) & = 3(g(x)) - 2                                  \\
                   & = 3\left(\frac{1}{3}x + \frac{2}{3}\right) - 2 \\
                   & = x + 2 - 2                                    \\
                   & = x
\end{align}

The second may done as follows:

\begin{align}
    (g \circ f)(x) & = \frac{1}{3}(f(x)) + \frac{2}{3}   \\
                   & = \frac{1}{3}(3x - 2) + \frac{2}{3} \\
                   & = x - \frac{2}{3} + \frac{2}{3}     \\
                   & = x
\end{align}

Then, we can see that the following statement holds true:

\begin{equation}
    (f \circ g)(x) = (g \circ f)(x) = x
\end{equation}

Note that will not hold true for any two functions, but rather will likely be false.
When the two compositions are the same and equal to $x$, it shows us that there is a nice relation between the two functions.
We will look at this relationship in a later section.

\end{document}