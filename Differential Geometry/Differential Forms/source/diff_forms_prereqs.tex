\documentclass[a4paper,12pt]{article}

\usepackage[left=25mm, right=25mm, top=25mm, bottom=25mm]{geometry}

\usepackage{amsmath}
\usepackage{amssymb}
\usepackage{amsfonts}
\usepackage[style=iso]{datetime2}
\usepackage[explicit]{titlesec}
\usepackage{amsthm}
\usepackage{mathrsfs}
\usepackage{array}
\usepackage{graphicx}
\usepackage{mathtools} % Provides \mathrlap command
\usepackage{xparse}
\usepackage{float}
\usepackage{fixdif}
\usepackage[skip=1em,indent]{parskip}
\usepackage{caption}

\usepackage{tocloft}
\usepackage{enumitem}
\usepackage[toc, page]{appendix}
\usepackage{accents}
\usepackage[perpage, symbol*]{footmisc} % for dagger footnotes
\DefineFNsymbols{sym}{\textdagger \textdaggerdbl \textsection \textparagraph %
{\textdagger\textdagger} {\textdaggerdbl\textdaggerdbl} %
{\textsection\textsection} {\textparagraph\textparagraph}}
\setfnsymbol{sym}
\interfootnotelinepenalty=10000

\usepackage{tikz}
\usepackage{tikz-3dplot}
\tikzset{>=latex} % for LaTeX arrow head
\usepackage{pgfplots} % for the axis environment
\usetikzlibrary{calc,decorations.markings}

\pgfplotsset{compat=1.18, every tick label/.append style={font=\footnotesize}}

\graphicspath{ {./Images/} }


% ---------------- %
\usepackage{etoolbox}
\usepackage[dvipsnames]{xcolor}
\usepackage[linkcolor=black]{hyperref}
\usepackage[nameinlink]{cleveref}

% Configure cleveref for appendix sections
\crefname{appsec}{Appendix}{appendices}
\Crefname{appsec}{Appendix}{Appendices}

\definecolor{RoyalRed}{RGB}{157, 16, 45}
\definecolor{RoyalOrange}{RGB}{249, 146, 69}

\makeatletter %* Creates a footnote colour option
\def\@footnotecolor{black}
\define@key{Hyp}{footnotecolor}{%
 \HyColor@HyperrefColor{#1}\@footnotecolor%
}
\patchcmd{\@footnotemark}{\hyper@linkstart{link}}{\hyper@linkstart{footnote}}{}{}
\makeatother

\hypersetup{
  colorlinks = true,
  linkcolor = RoyalRed,
  citecolor = RoyalOrange,
  footnotecolor = black,
}
% ---------------- %
\makeatletter
\def\th@plain{%
  \thm@notefont{}% same as heading font
  \itshape % body font
}
\def\th@definition{%
  \thm@notefont{}% same as heading font
  \normalfont % body font
}
\makeatother

\newcommand*\diff{\mathop{}\!d} % for the differential in integrals
% ---------------- %
\ExplSyntaxOn %* This command is for row vectors; unstarred version has no delimiter scaling for inline; starred version has delimiter scaling for display
\seq_new:N \l_user_rvec_seq

% expl3 worker (does the splitting/formatting)
\cs_new_protected:Npn \dbacc_rvec:n #1
 {
  \seq_clear:N \l_user_rvec_seq
  \seq_set_split:Nnn \l_user_rvec_seq { , } { #1 }
  \seq_use:Nn \l_user_rvec_seq { \enspace }
 }

% plain-name wrapper (no ':' in its name) — safe to call outside ExplSyntaxOn
\cs_new_protected:Npn \dbaccrvec #1 { \dbacc_rvec:n { #1 } }

% user-level \rvec interface (star + optional size + mandatory arg)
\NewDocumentCommand \rvec { s o m }
 {
  \IfBooleanTF{#1}
    { \left[\,\dbaccrvec{#3}\,\right] } % starred -> automatic \left...\right
    { \IfValueTF{#2}
        { \mathopen{#2[}\,\dbaccrvec{#3}\,\mathclose{#2]} } % sized optional arg
        { [\,\dbaccrvec{#3}\,] } % default small delimiters
    }
 }
\ExplSyntaxOff
% ---------------- %
\newcommand{\R}{\mathbb{R}}
\newcommand{\C}{\mathbb{C}}
\newcommand{\Z}{\mathbb{Z}}

\NewCommandCopy{\oldIm}{\Im}
\renewcommand{\Im}{\mathop{\oldIm\mathfrak{m}}}
\NewCommandCopy{\oldRe}{\Re}
\renewcommand{\Re}{\mathop{\oldRe\mathfrak{e}}}

\newcommand{\overbar}[1]{\mkern 1.5mu\overline{\mkern-1.5mu#1\mkern-1.5mu}\mkern 1.5mu}

\DeclareMathOperator{\Res}{Res}

\DeclarePairedDelimiterX{\norm}[1]{\lVert}{\rVert}{#1}
\DeclarePairedDelimiterX{\abs}[1]{\lvert}{\rvert}{#1}
\NewDocumentCommand{\eval}{smm}{%
  \IfBooleanTF{#1}
    {#2\rvert_{#3}}
    {\left.#2\right\rvert_{#3}}%
}
% ---------------- %
\newcommand{\ihat}{\boldsymbol{\hat{\imath}}}
\newcommand{\jhat}{\boldsymbol{\hat{\jmath}}}
\newcommand{\khat}{\boldsymbol{\hat{k}}}
\newcommand{\ehat}{\mathbf{\hat{e}}}
\newcommand{\T}{\mathrm{T}}
% ---------------- %

\newtheorem{theorem}{Theorem}
\newtheorem{lemma}[theorem]{Lemma}
\newtheorem{proposition}[theorem]{Proposition}
\newtheorem{corollary}{Corollary}[theorem]

\theoremstyle{definition}
\newtheorem{definition}[theorem]{Definition}
\newtheorem{example}[theorem]{Example}

% \setlength{\cftsecnumwidth}{3em}

\begin{titlepage}
\title{Differential Forms Prerequisites (ROUGH)}
\author{Abdul Musthakin}
\date{November 2025}
\end{titlepage}

% \renewcommand{\thesection}{\Roman{section}}

\allowdisplaybreaks

\setlength{\parindent}{0pt}

\begin{document}
\maketitle

\section{Definition of a Vector} \label{appendix:vector_def}

An introduction to the concept of vectors usually involve giving examples of vectors -- such as arrows in a plane -- without giving a proper definition of what a vector is in general.
That makes sense, since such a definition is (relatively) technical and unnecessary for simply working with vectors.
The technical definition is that a vector is an element of a vector space, but that alone simply replaces one question with another.
It follows that a definition of a vector space is now required.

To come up with it, we will follow a method of generalization that can be readily applied to other situations as well.
We know that ordered pairs of real numbers are examples of vectors, so the set containing all such pairs is an example of a vector space.\footnote{This is technically an abuse of language. A vector space is a set with additional structure -- the ability to add and scale its elements in a certain way. However, this distinction is not very useful until we actually define a vector space properly.}
This set is denoted by $\R \times \R$, and $\R^2$ is a useful shorthand.
We will examine the properties of this set in order to obtain a general definition of a vector space.

Two things we can do with our vectors is add them together and scale them.
For the example we are working with, this is done component-wise: $(a, b) + (b, c) = (a + b, c + d)$ and $c \cdot (a, b) = (ac, bc)$.
Since we can use real numbers to scale vectors, they are also referred to as scalars in this context.
Both of those functions have geometric interpretations if the vectors are viewed as arrows on a plane.
Vector addition involves connecting two vectors tip-to-tail and the sum is the third side of a triangle (or the diagonal of a parallelogram).
Scalar multiplication involved literally scaling the length of a vector by some number, where multiplication by a negative number reverses direction.

Whilst vectors in general are not arrows, the way they add and scale is fundamental to them.
If we try to extract all the properties of addition and scalar multiplication, we get the following list.
Note that the boldface letters are elements of $\R^2$, whilst the greek letters are real numbers.
Also, the boldface zero represents (0, 0).
\begin{enumerate}
\item $\mathbf{u} + \mathbf{v} = \mathbf{v} + \mathbf{u}$
\item $\mathbf{u} + (\mathbf{v} + \mathbf{w}) = (\mathbf{u} + \mathbf{v}) + \mathbf{w}$
\item $\mathbf{0} + \mathbf{v} = \mathbf{v}$
\item $-\mathbf{v} + \mathbf{v} = \mathbf{0}$
\item $\lambda(\mu\mathbf{v}) = (\lambda\mu)\mathbf{v}$
\item $1\mathbf{v} = \mathbf{v}$
\item $\lambda(\mathbf{v} + \mathbf{w}) = \lambda\mathbf{v} + \lambda\mathbf{w}$
\item $(\lambda + \mu)\mathbf{v} = \lambda\mathbf{v} + \mu\mathbf{v}$
\end{enumerate}
That leads to our definition of a vector space -- a set, together with vector addition and scalar multiplication, that satisfies the above statements.\footnote{Addition and scalar multiplication must be closed, but this is within their definitions, rather than being a part of the axioms.}
More precisely, the third and fourth statements say that an identity vector (that works for all vectors) and inverse vectors (that each work for a specific vector) exist.
These statements are known as the vector space axioms.
Note that choosing those specific axioms to define a vector space is just that: a choice.\footnote{It is not an arbitrary choice, however. Vectors are an incredibly useful concept both in pure maths and in other subjects -- where they are applied (along with linear algebra in general) perhaps more than any other topic in maths.}
We can explore sets with different strucutres, obtains objects like groups and rings instead.

A common notation for vectors is boldface letters, and that is what will be used in this document.
Regular real numbers are never in boldface, though they might use either the Latin or Greek alphabet.
Additionally, whilst there are many different objects that classify as vectors, the only ones that will be considered by us are in fact lists of real numbers.
An ordered listed of $n$ real numbers is an element of $\R^n$.
In this context, $n$ is always interpreted as being a positive integer.

The above contains sufficient information to define vectors and work with them on a basic level.
What follows are some additional points of discussion.
Firstly, there are many different notations that exist for vectors.
Arrows, particularly in physics, can also be used to denote vectors, e.g. $\vec{F} = m\vec{a}$.
It is not easy to write boldface symbols in handwriting, so another alternative is underlining the vector, $\underline{v}$, or putting a tilde beneath it, $\underaccent{\sim}{v}$.
For the actual value of the vector, you can write it as a coordinate, as was shown before.
Another common way is to write it as a column of numbers.
\begin{equation*}
    \mathbf{v} = \begin{bmatrix}
        a \\
        b
    \end{bmatrix}
\end{equation*}
It can be written also be written like as $\mathbf{v} = \rvec{a,b}$. Note that the former is called a column vector, whilst the latter is a row vector.
Technically, they are different objects, but this distinction only matters when dealing with matrices.
Another way to write a vector is $\mathbf{v} = \langle a, b \rangle$.
This might be preferred due to avoiding confusion with coordinates, though writing $\mathbf{v} = (a,b)$ is pretty common.
Additionally, given that $\ihat = \langle 1, 0 \rangle$ and $\jhat = \langle 0, 1 \rangle$, you can write a vector as $\mathbf{v} = v_x\ihat + v_y\jhat$, where $v_x$ and $v_y$ are the components of the vector.
Note that these examples were given for $\R^2$, but they extend to general vectors in $\R^n$.

Another point to make is that vector addition is a binary operation.
It takes two vectors as an input and outputs another vector.
On the other hand, scalar multiplication is a binary function.
It takes a vector and a scalar as an input and outputs a vector.
All operations are functions, but the converse is not true in general.
An operation is a function whose domain is some power of the codomain.
For example, vector addition -- in reference to elements of $\R^n$ -- may be written as $+\colon \R^{2n} \to \R^n$.
This is a matter of semantics and using the word operation loosely is both common and unharmful.

\section{Matrices} \label{appendix:matrices}

Matrices, unlike vectors, are relatively easy to define.
An $m \times n$ matrix is a rectangular array of objects with $m$ rows and $n$ columns.
Each object in the matrix is a particular entry in, or component of the matrix.
The usual notation for a variable that is a matrix is boldface upper-case letters, e.g. $\mathbf{A}$.
Below is an example of a matrix containing real numbers.
\begin{equation*}
    \mathbf{A} = \begin{bmatrix}
        1 & 3 & 0.4 \\
        2 & 9 & 0
    \end{bmatrix}
\end{equation*}
In this case, the size or order of $\mathbf{A}$ is $2 \times 3$.
We may use either round or square brackets for a matrix.
The entry in the $i$-th row and $j$-th column may be referenced as $A_{ij}$, $\mathbf{A}_{ij}$, or $a_{ij}$, with or without a comma separating $i$ and $j$.
This means that we can write a general $m \times n$ matrix as
\begin{equation*}
    \mathbf{A} = \begin{bmatrix}
        A_{11} & A_{12} & \cdots & A_{1n} \\
        A_{21} & A_{22} & \cdots & A_{2n} \\
        \vdots & \vdots & \ddots & \vdots \\
        A_{m1} & A_{m2} & \cdots & A_{mn}
    \end{bmatrix}.
\end{equation*}
The entry $A_{ij}$ may also be called the $(i, j)$ entry of the matrix.
A matrix is uniquely determined by its order and a formula for a generic term.
We may specify a matrix $\mathbf{A}$ with its generic entry by writing $\mathbf{A} = (A_{ij})$, and possibly specifying the range of $i$ and $j$.
If $m = n$, the matrix is called a square matrix.
If $m = 1$ or $n = 1$, then the matrix is called a column or row matrix/vector respectively.

The actual purpose of matrices only comes when we introduce the operations we can perform on them.
We can add two matrices by adding their components: $(\mathbf{A} + \mathbf{B})_{ij} = A_{ij} + B_{ij}$.
Also, we can scale matrices by multiplying their components by some number: $(c\mathbf{A})_{ij} = cA_{ij}$.
In this way, matrices work similarly to vectors.
However, we are also able to multiply matrices together.
Simply providing a formula would not be very enlightening, so we will first introduce the notion that matrices are linear maps.

Consider the $n$-dimensional vector $\mathbf{x}$.
We can express the vector a sequence of scalars and construct an $n \times 1$ matrix with them.
This can be done for any vector in a finite-dimensional vector space.
We can multiply a general matrix by this vector as follows.
\begin{equation*}
    \mathbf{Ax} = \begin{bmatrix}
        A_{11} & A_{12} & \cdots & A_{1n} \\
        A_{21} & A_{22} & \cdots & A_{2n} \\
        \vdots & \vdots & \ddots & \vdots \\
        A_{m1} & A_{m2} & \cdots & A_{mn}
    \end{bmatrix}
    \begin{bmatrix}
        x_1 \\
        x_2 \\
        \vdots \\
        x_n
    \end{bmatrix}
    =
    \begin{bmatrix}
        A_{11}x_1 + A_{12}x_2 + \cdots + A_{1n}x_n \\
        A_{21}x_1 + A_{22}x_2 + \cdots + A_{2n}x_n \\
        \vdots \\
        A_{m1}x_1 + A_{m2}x_2 + \cdots + A_{mn}x_n \\
    \end{bmatrix}
\end{equation*}
This is transforming each component of the vector by taking the dot product of the corresponding row of the matrix with the vector.
It can thus be seen that a matrix maps one vector to another vector.
The linear part comes from the fact that $\mathbf{A}(\lambda\mathbf{x} + \mu\mathbf{y}) = \lambda\mathbf{Ax} + \mu\mathbf{Ay}$, which follows from the way matrices add and scale.
Then, the interpretation of multiplying two matrices $\mathbf{A}$ and $\mathbf{B}$ together is creating a composite transformation, similar to how two functions can be composed.
The general formula for this is stil not particularly enlightening, but it is given below.
\begin{equation*}
    (\mathbf{AB})_{ij} = A_{i1}B_{1j} + A_{i2}B_{2j} + \cdots + A_{in}B_{nj} = \sum_{k=1}^n A_{ik}B_{kj}
\end{equation*}
Here, $n$ is the number of columns of the first matrix and the number of rows of the second.
Multiplying two matrices (in this way) is only defined when these two numbers are equal.
If you multiply an $m \times n$ matrix by a $p \times q$ matrix, the order of the product will be $m \times q$.

From the fact that matrix multiplication represents composing transformations, it follows that matrix multiplication is associative: $(\mathbf{AB})\mathbf{C} = \mathbf{A}(\mathbf{BC}) = \mathbf{ABC}$.
It is not commutatve however, meaning that $\mathbf{AB} \neq \mathbf{BA}$.
In general, the order in which transformations are applied matter -- rotating then reflecting an object is not the same thing as reflecting and then rotating it.
The order in which transformations are applied is right to left (from our definition).

An $m \times 1$ matrix (row vector) and a $1 \times n$ matrix (column vector) can be associated with each other, and they can both be associated with an $n$-dimensional vector.
Their difference comes from the way they multiply with other matrices.
This can be seen from the definition and the various rules that come out of it.
We developed matrix multiplication by treating a vector as a $m \times 1$ matrix and multiplying by a matrix on the left.
Treating it instead as a $1 \times n$ matrix and multiplying by a matrix on the right would have given the same transformation (but for a row vector).

Multiplying a column vector by a row vector will yield a $1 \times 1$ matrix.
In general, one can associate any $1 \times 1$ matrix with a scalar; note that they are different objects, though (just as matrices with a single row or column are technically different from vectors).
This association allows for one to define the dot product of two vectors as the product of a row vector and a column vector.
When doing so, the column vector is written as the transpose of a row vector: $\mathbf{x} \cdot \mathbf{y} = \mathbf{x}\mathbf{y}^\T$.
The transpose of a matrix in general is defined by the simple rule $(\mathbf{A})_{ij} = A_{ji}$, meaning the rows and columns are swapped around.
