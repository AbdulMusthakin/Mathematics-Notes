\documentclass[a4paper]{article}

\usepackage{amsmath}
\usepackage{amssymb}
\usepackage{amsfonts}
\usepackage[style=iso]{datetime2}
\usepackage[explicit]{titlesec}
\usepackage{amsthm}
\usepackage{mathrsfs}
\usepackage{array}
\usepackage{graphicx}
\usepackage{mathtools} % Provides \mathrlap command
\usepackage{float}
\usepackage[skip=1em,indent]{parskip}
\usepackage{caption}
\usepackage{hyperref}
\usepackage{tocloft}
\usepackage{mathtools}
\usepackage{dsfont}

% \usepackage{tikz}
% \tikzset{>=latex} % for LaTeX arrow head
% \usepackage{pgfplots} % for the axis environment
% \usetikzlibrary{calc,decorations.markings}

% \pgfplotsset{compat=1.18, every tick label/.append style={font=\footnotesize}}

\graphicspath{ {./Images/} }

\makeatletter
\def\th@plain{%
  \thm@notefont{}% same as heading font
  \itshape % body font
}
\def\th@definition{%
  \thm@notefont{}% same as heading font
  \normalfont % body font
}
\makeatother

\newcommand*\diff{\mathop{}\!d} % for the differential in integrals

\newtheorem{theorem}{Theorem}
\newtheorem{lemma}[theorem]{Lemma}
\newtheorem{proposition}[theorem]{Proposition}
\newtheorem{corollary}{Corollary}[theorem]

\theoremstyle{definition}
\newtheorem{definition}{Definition}
\newtheorem*{example}{Example}

% \setlength{\cftsecnumwidth}{3em}

\begin{titlepage}
\title{Arithmetic Functions}
\author{Abdul Musthakin}
\date{\today}
\end{titlepage}

% \renewcommand{\thesection}{\Roman{section}}

\allowdisplaybreaks

\setlength{\parindent}{0pt}

\begin{document}
\maketitle

\section{Introduction}

\begin{definition}[Arithmetic Function]
    A function $f: X \to Y$ is an arithmetic function iff its domain is the set of positive integers: $X = \mathbb{N} \coloneq \mathbb{Z}_{>0}$.
\end{definition}
These functions are a major subject of study in number theory, and are often referred to as \textbf{number-theoretic} functions, though that may also be used to refer to any function that has use within number theory.
First, we will define some properties that an arithemtic function can have.
\begin{definition}
    An arithemtic function $f$ is additive if $f(mn) = f(m) + f(n)$ for all $m, n \in \mathbb{N}$ such that $\gcd(m,n)=1$.
    Furthermore, it is completely additive if $f(mn) = f(m) + f(n)$ for all $m, n \in \mathbb{N}$.
\end{definition}
\begin{definition}
    An arithemtic function $f$ is multiplicative if $f(mn) = f(m)f(n)$ for all $m, n \in \mathbb{N}$ such that $\gcd(m,n)=1$.
    Furthermore, it is completely multiplicative if $f(mn) = f(m)f(n)$ for all $m, n \in \mathbb{N}$.
\end{definition}

\section{Divisor Functions}

The first arithmetic functions that we will discuss are the divisor functions, which are an infinite family of functions (though two are of particular importance).
\begin{definition}[Divisor Function]
    The sum of divisors functions, $\sigma_z: \mathbb{N} \to \mathbb{R}$, for some $z \in \mathbb{R}$, are arithemtic functions given by
    \begin{equation}
        \sigma_z(n) \coloneq \sum_{d \mid n} d^z.
    \end{equation}
    In particular, $\tau(n) \coloneq \sigma_0(n)$ and $\sigma(n) \coloneq \sigma_1(n)$.
\end{definition}
We can see that $\tau(n)$ counts the divisors/factors of $n$, whilst $\sigma(n)$ sums them.
\begin{example} \leavevmode
    \begin{itemize}
        \item Since the factors of 6 are 1, 2, 3, and 6, $\tau(6) = 4$ and $\sigma(6) = 12$.
        \item For any prime $p$, $\tau(p)$ = 2 and $\sigma(p) = p + 1$.
    \end{itemize}
\end{example}
It turns out that the divisor functions are multiplicative, but not completely multiplicative.
The latter can be shown rather easily via counterexamples.
We will prove the former result for $\tau$ and $\sigma$ specifically.
\begin{proposition} \label{thm:Prop 1}
    The function $\tau$ is multiplicative.
\end{proposition}
\begin{proof}
    Let $m,n \in \mathbb{N}$ such that $\gcd(m,n) = 1$.
    Let the function $D: \mathbb{N} \to \mathcal{P}(\mathbb{N})$ be given by $D(n) \coloneq \{x \in \mathbb{N} \mathrel{:} x \mid n\}$. Here, $D(n)$ is the set of all divisors of $n$, meaning that $\tau(n) = |D(n)|$.
    Proving that $\tau$ is multiplicative is equivalent to proving that there is a bijection between $D(m) \times D(n)$ and $D(mn)$, which will be shown at the end.

    Let $p \in D(m)$ and $q \in D(n)$.
    Consider the function $f: D(m) \times D(n) \to D(mn)$ given by $f(p,q) \coloneq pq$.
    To show that this actually is a function, we must prove that $pq \in D(mn)$.
    We have the implications $p \in D(m) \iff (\exists k \in \mathbb{N})(m = kp)$ and $q \in D(n) \iff (\exists l \in \mathbb{N})(n = lq)$.
    Hence, $mn = klpq$, where $kl \in \mathbb{N}$.
    Thus, $pq \mid mn$, so $pq \in D(mn)$.

    Now, we have to show that $f$ is injective.
    Let $a,c \in D(m)$ and $b,d \in D(n)$ such that $f(a,b) = f(c,d)$.
    That means $ab = cd$, giving $a \mid cd$ and $c \mid ab$.
    Since $\gcd(m,n) = 1$, we get $\gcd(a,d) = \gcd(b,c) = 1$.
    By \textbf{Euclid's Lemma}, since $a \mid cd$ and $\gcd(a,d)=1$, we get $a \mid c$.
    Similarly, $c \mid a$, but that must mean that $a=c$, and hence $b=d$.
    Thus, $f(a,b) = f(c,d) \implies (a,b) = (c,d)$.

    It remains to be proved that $f$ is surjective.
    Let $d \in D(mn)$, $x \coloneq \gcd(d,m)$ and $y \coloneq \gcd(d,n)$.
    It follows that $x \in D(m)$ and $y \in D(n)$.
    Then,
    \begin{align*}
        xy & = \gcd(d,m) \gcd(d,n)                              \\
           & = \gcd(mn, dm, dn, dd)   & (\text{gcd laws})       \\
           & = \gcd(mn, \gcd(m,n,d)d) & (\text{distributivity}) \\
           & = \gcd(mn, d)            & \because \gcd (m,n)=1   \\
           & = d.
    \end{align*}
    As the function $f$ is injective and surjective, $f$ is bijective.
    We thus get $f: D(m) \times D(n) \leftrightarrow D(mn) \implies |D(m) \times D(n)| = |D(mn)|$.
    Therefore, $\tau(mn) = |D(mn)| = |D(m) \times D(n)| = |D(m)||D(n)| = \tau(m)\tau(n)$.
\end{proof}
\begin{proposition}
    The function $\sigma$ is multiplicative.
\end{proposition}
\begin{proof}
    Consider the proof of Proposition~\ref{thm:Prop 1} and all of its definitions.
    The main fact to note is that there is a bijection between $D(m) \times D(n)$ and $D(mn)$.
    It thus follows that
    \begin{equation*}
        \sigma(m)\sigma(n) = \left( \sum_{d_1 \mid m} d_1 \right) \left( \sum_{d_2 \mid n} d_2 \right) = \sum_{\substack{d_1 \mid m \\ d_2 \mid n}} d_1 d_2 = \sum_{k \mid mn} k = \sigma(mn).
    \end{equation*}
\end{proof}
By the \textbf{fundamental theorem of arithmetic}, we can express a number $n \in \mathbb{N}$ as a product of primes, which is unique up to the order of the factors.
\begin{equation*}
    n = \prod_{i=1}^r p_i^{a_i},
\end{equation*}
where $r$ is the number of distinct prime factors of $n$, $p_i$ is the $i$-th prime factor, and $a_i = \max \{x \in \mathbb{N} \mathrel{:} p_i^x \mid n \}$.
Since $\tau$ is multiplicative,
\begin{equation}
    \tau(n) = \tau \left( \prod_{i=1}^r p_i^{a_i} \right) = \prod_{i=1}^r \tau (p_i^{a_i}) = \prod_{i=1}^r (a_i + 1).
\end{equation}
Similarly, as $\sigma$ is multiplicative,
\begin{align*}
    \sigma(n) & = \sigma \left( \prod_{i=1}^r p_i^{a_i} \right)                                           \\
              & = \prod_{i=1}^r \sigma (p_i^{a_i})                                                        \\
              & = \prod_{i=1}^r \sum_{j=0}^{a_i} p_i^j                                                    \\
              & = \prod_{i=1}^r \frac{p_i^{a_i+1} - 1}{p_i - 1}, \stepcounter{equation}\tag{\theequation}
\end{align*}
with the final result being obtained using the formula for a finite geometric series.

\section{Prime Counting Function}

\begin{definition}[Prime Counting Function]
    The function $\pi: \mathbb{R}_{>0} \to \mathbb{N}$ is a number-theoretic (but not arithemtic) function given by
    \begin{equation*}
        \pi(x) \coloneq \sum_{n \leq x} \mathds{1}_\mathbb{P},
    \end{equation*}
    where $x, n \in \mathbb{R}_{>0}$ and $\mathds{1}_\mathbb{P}$ is the characteristic function for the set of prime numbers.
\end{definition}
\end{document}