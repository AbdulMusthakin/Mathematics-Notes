\documentclass[a4paper]{article}

\usepackage{amsmath}
\usepackage{amssymb}
\usepackage{amsfonts}
\usepackage[style=iso]{datetime2}
\usepackage[explicit]{titlesec}
\usepackage{amsthm}
\usepackage{mathrsfs}
\usepackage{array}
\usepackage{graphicx}
\usepackage{mathtools} % Provides \mathrlap command
\usepackage{float}
\usepackage[skip=1em,indent]{parskip}
\usepackage{caption}
\usepackage{hyperref}
\usepackage{tocloft}
\usepackage{mathtools}

\usepackage{scalerel}
\newcommand{\bbone}{\scalerel*{\text{\usefont{U}{bbold}{m}{n}1}}{\mathbb{R}}}
\MakeRobust{\bbone}

\DeclareMathOperator{\lcm}{lcm}
\DeclareMathOperator{\im}{im}

% \usepackage{tikz}
% \tikzset{>=latex} % for LaTeX arrow head
% \usepackage{pgfplots} % for the axis environment
% \usetikzlibrary{calc,decorations.markings}

% \pgfplotsset{compat=1.18, every tick label/.append style={font=\footnotesize}}

\graphicspath{ {./Images/} }

\makeatletter
\def\th@plain{%
  \thm@notefont{}% same as heading font
  \itshape % body font
}
\def\th@definition{%
  \thm@notefont{}% same as heading font
  \normalfont % body font
}
\makeatother

\newcommand*\diff{\mathop{}\!d} % for the differential in integrals

\newtheorem{theorem}{Theorem}
\newtheorem{lemma}[theorem]{Lemma}
\newtheorem{proposition}[theorem]{Proposition}
\newtheorem{corollary}{Corollary}[theorem]

\theoremstyle{definition}
\newtheorem{definition}{Definition}
\newtheorem*{example}{Example}

% \setlength{\cftsecnumwidth}{3em}

\begin{titlepage}
\title{Arithmetic Functions}
\author{Abdul Musthakin}
\date{\today}
\end{titlepage}

% \renewcommand{\thesection}{\Roman{section}}

\allowdisplaybreaks

\setlength{\parindent}{0pt}

\begin{document}
\maketitle

\section{Introduction}

\begin{definition}[Arithmetic Function]
    A function $f\colon X \to Y$ is an arithmetic function iff its domain is the set of positive integers: $X = \mathbb{N} \coloneq \mathbb{Z}_{>0}$.
\end{definition}
These functions are a major subject of study in number theory, and are often referred to as \textbf{number-theoretic} functions, though that may also be used to refer to any function that has use within number theory.
First, we will define some properties that an arithemtic function can have.
\begin{definition}
    An arithemtic function $f$ is additive if $f(mn) = f(m) + f(n)$ for all $m, n \in \mathbb{N}$ such that $\gcd(m,n)=1$.
    Furthermore, it is completely additive if $f(mn) = f(m) + f(n)$ for all $m, n \in \mathbb{N}$.
\end{definition}
\begin{definition}
    An arithemtic function $f$ is multiplicative if $f(mn) = f(m)f(n)$ for all $m, n \in \mathbb{N}$ such that $\gcd(m,n)=1$.
    Furthermore, it is completely multiplicative if $f(mn) = f(m)f(n)$ for all $m, n \in \mathbb{N}$.
\end{definition}

\section{Divisor Functions}

The first arithmetic functions that we will discuss are the divisor functions, which are an infinite family of functions (though two are of particular importance).
\begin{definition}[Divisor Function]
    The sum of divisors functions, $\sigma_z\colon \mathbb{N} \to \mathbb{R}$, for some $z \in \mathbb{R}$, are arithemtic functions given by
    \begin{equation}
        \sigma_z(n) \coloneq \sum_{d \mid n} d^z.
    \end{equation}
    In particular, $\tau(n) \coloneq \sigma_0(n)$ and $\sigma(n) \coloneq \sigma_1(n)$.
\end{definition}
We can see that $\tau(n)$ counts the divisors/factors of $n$, whilst $\sigma(n)$ sums them.
\begin{example} \leavevmode
    \begin{itemize}
        \item Since the factors of 6 are 1, 2, 3, and 6, $\tau(6) = 4$ and $\sigma(6) = 12$.
        \item For any prime $p$, $\tau(p)$ = 2 and $\sigma(p) = p + 1$.
    \end{itemize}
\end{example}
It turns out that the divisor functions are multiplicative, but not completely multiplicative.
The latter can be shown rather easily via counterexamples.
We will prove the former result for $\tau$ and $\sigma$ specifically.
\begin{proposition} \label{thm:tau multiplicative}
    The function $\tau$ is multiplicative.
\end{proposition}
\begin{proof}
    Let $m,n \in \mathbb{N}$ such that $\gcd(m,n) = 1$.
    Let the function $D\colon \mathbb{N} \to \mathcal{P}(\mathbb{N})$ be given by $D(n) \coloneq \{x \in \mathbb{N} : x \mid n\}$. Here, $D(n)$ is the set of all divisors of $n$, meaning that $\tau(n) = |D(n)|$.
    Proving that $\tau$ is multiplicative is equivalent to proving that there is a bijection between $D(m) \times D(n)$ and $D(mn)$, which will be shown at the end.

    Let $p \in D(m)$ and $q \in D(n)$.
    Consider the function $f\colon D(m) \times D(n) \to D(mn)$ given by $f(p,q) \coloneq pq$.
    To show that this actually is a function, we must prove that $pq \in D(mn)$.
    We have the implications $p \in D(m) \iff (\exists k \in \mathbb{N})(m = kp)$ and $q \in D(n) \iff (\exists l \in \mathbb{N})(n = lq)$.
    Hence, $mn = klpq$, where $kl \in \mathbb{N}$.
    Thus, $pq \mid mn$, so $pq \in D(mn)$.

    Now, we have to show that $f$ is injective.
    Let $a,c \in D(m)$ and $b,d \in D(n)$ such that $f(a,b) = f(c,d)$.
    That means $ab = cd$, giving $a \mid cd$ and $c \mid ab$.
    Since $\gcd(m,n) = 1$, we get $\gcd(a,d) = \gcd(b,c) = 1$.
    By \textbf{Euclid's Lemma}, since $a \mid cd$ and $\gcd(a,d)=1$, we get $a \mid c$.
    Similarly, $c \mid a$, but that must mean that $a=c$, and hence $b=d$.
    Thus, $f(a,b) = f(c,d) \implies (a,b) = (c,d)$.

    It remains to be proved that $f$ is surjective.
    Let $d \in D(mn)$, $x \coloneq \gcd(d,m)$ and $y \coloneq \gcd(d,n)$.
    It follows that $x \in D(m)$ and $y \in D(n)$.
    Then,
    \begin{align*}
        xy & = \gcd(d,m) \gcd(d,n)                              \\
           & = \gcd(mn, dm, dn, dd)   & (\text{gcd laws})       \\
           & = \gcd(mn, \gcd(m,n,d)d) & (\text{distributivity}) \\
           & = \gcd(mn, d)            & \because \gcd (m,n)=1   \\
           & = d.
    \end{align*}
    As the function $f$ is injective and surjective, $f$ is bijective.
    We thus get $f\colon D(m) \allowbreak \times D(n) \leftrightarrow D(mn) \implies |D(m) \times D(n)| = |D(mn)|$.
    Therefore, $\tau(mn) = |D(mn)| = |D(m) \times D(n)| = |D(m)||D(n)| = \tau(m)\tau(n)$.
\end{proof}
\begin{proposition}
    The function $\sigma$ is multiplicative.
\end{proposition}
\begin{proof}
    Consider the proof of Proposition~\ref{thm:tau multiplicative} and all of its definitions.
    The main fact to note is that there is a bijection between $D(m) \times D(n)$ and $D(mn)$.
    It thus follows that
    \begin{equation*}
        \sigma(m)\sigma(n) = \left( \sum_{d_1 \mid m} d_1 \right) \left( \sum_{d_2 \mid n} d_2 \right) = \sum_{\substack{d_1 \mid m \\ d_2 \mid n}} d_1 d_2 = \sum_{k \mid mn} k = \sigma(mn).
    \end{equation*}
\end{proof}
By the \textbf{fundamental theorem of arithmetic}, we can express a number $n \in \mathbb{N}$ as a product of primes, which is unique up to the order of the factors.
\begin{equation*}
    n = \prod_{i=1}^r p_i^{a_i},
\end{equation*}
where $r$ is the number of distinct prime factors of $n$, $p_i$ is the $i$-th prime factor, and $a_i = \max \{x \in \mathbb{N} : p_i^x \mid n \}$.
Since $\tau$ is multiplicative,
\begin{equation}
    \tau(n) = \tau \left( \prod_{i=1}^r p_i^{a_i} \right) = \prod_{i=1}^r \tau (p_i^{a_i}) = \prod_{i=1}^r (a_i + 1).
\end{equation}
Similarly, as $\sigma$ is multiplicative,
\begin{align*}
    \sigma(n) & = \sigma \left( \prod_{i=1}^r p_i^{a_i} \right)                                           \\
              & = \prod_{i=1}^r \sigma (p_i^{a_i})                                                        \\
              & = \prod_{i=1}^r \sum_{j=0}^{a_i} p_i^j                                                    \\
              & = \prod_{i=1}^r \frac{p_i^{a_i+1} - 1}{p_i - 1}, \stepcounter{equation}\tag{\theequation}
\end{align*}
with the final result being obtained using the formula for a finite geometric series.

\section{Prime Counting Function}

\begin{definition}[Prime Counting Function]
    The function $\pi\colon \mathbb{R}_{>0} \to \mathbb{N}$ is a number-theoretic (but not arithemtic) function given by
    \begin{equation}
        \pi(x) \coloneq \sum_{n \leq x} \bbone_\mathbb{P} (n),
    \end{equation}
    where $x, n \in \mathbb{R}_{>0}$ and $\bbone_\mathbb{P}$ is the characteristic function for the set of prime numbers.
\end{definition}

\section{M\"obius Function}

\begin{definition}[M\"obius Function]
    The function $\mu\colon \mathbb{N} \to \{-1, 0, 1\}$ is a arithemtic function given by
    \begin{equation}
        \mu(n) \coloneq
        \begin{cases}
            1      & \text{if } n = 1,                                            \\
            (-1)^k & \text{if } n \text{ is a product of $k$ distinct primes,}    \\
            0      & \text{if } n \text{ is divisble by a square greater than 1.}
        \end{cases}
    \end{equation}
\end{definition}
The M\"obius function has some important properties.
First, we will show that it is a multiplicative function.
\begin{proposition}
    The M\"obius function is multiplicative.
\end{proposition}
\begin{proof}
    Let $m,n \in \mathbb{N}$ such that $\gcd(m,n) = 1$ and, without loss of generality, $m \geq n$.
    Additionally, let $m$ and $n$ have the following prime factorisations.
    \begin{equation*}
        m \coloneq \prod_{i=1}^r p_i^{a_i} \qquad n \coloneq \prod_{j=1}^s q_j^{b_j}
    \end{equation*}
    If $m = n = 1$, then $\mu(mn) = 1 = \mu(m)\mu(n)$ since $\mu(1) = 1$.
    Otherwise, $m > n \geq 1$.
    If $n = 1$, then $\mu(mn) = \mu(m) = \mu(m)\mu(n)$ since $\mu(n) = 1$.
    Otherwise, $m > n > 1$.
    If there exists any $a_i$ or $b_j$ greater than 1, then $\mu(mn) = 0$, and we have $\mu(m) = 0$ or $\mu(n) = 0$.
    Otherewise, $\forall i, j\colon a_i = b_j = 1$.
    Then, $\mu(mn) = (-1)^{r+s} = (-1)^r \cdot (-1)^s = \mu(m)\mu(n)$.
\end{proof}
There is another property of the M\"obius function worth noting, which has to do with summing the function evaluated at the factors of some number.
\begin{proposition}
    For all $n \in \mathbb{N}$,
    \begin{equation}
        \sum_{d \mid n} \mu (d) = \begin{cases}
            1 & \text{if } n = 1, \\
            0 & \text{if } n > 1.
        \end{cases}
    \end{equation}
\end{proposition}
\begin{proof}
    Let $n \in \mathbb{N}$ with the following prime factorisation.
    \begin{equation*}
        n \coloneq \prod_{i=1}^r p_i^{a_i}
    \end{equation*}
    Any factor $d$ will be of the form
    \begin{equation*}
        d \coloneq \prod_{i=1}^r p_i^{b_i},
    \end{equation*}
    where $\forall i\colon b_i \leq a_i$.
    If $n = 1$, then the sum is simply $\mu(1) = 1$.
    Otherwise, $n > 1$.
    The only nonzero terms in the sum are those for which $d$ is a product of distinct primes; $d=1$ can be considered as the case of zero primes.
    All possible values of $d$ can be arranged in the following way: $1, \quad p_1, p_2, \dots, p_r, \quad p_1 p_2, p_1 p_3, \dots, p_1 p_r, \allowbreak \quad \dots, \quad p_1 p_2 \cdots p_r$.
    It can thus be seen that there are $\binom{r}{i}$ factors of n that are the products of $i$ distinct primes.
    Hence,
    \begin{equation*}
        \sum_{d \mid n} \mu (d) = \sum_{i=0}^{r} \binom{r}{i} (-1)^i = 0,
    \end{equation*}
    as this is the binomial expansion of $(1 + x)^i$ for $x=-1$.
\end{proof}

\section{Euler's Totient Function}

\begin{definition}[Euler's Totient Function]
    The function $\varphi\colon \mathbb{N} \to \mathbb{N}$ is an arithemtic function given by
    \begin{equation}
        \varphi(n) \coloneq \sum_{\mathclap{\substack{i=1;\\ \gcd(i,n)=1}}}^n \; 1.
    \end{equation}
\end{definition}
This function -- alternatively called Euler's phi function -- evaluated at $n \in \mathbb{N}$, counts the natural numbers up to $n$ that are coprime to $n$.
This function is multiplicative, which we will show in two different ways, though we need to prove a lemma first.
\begin{lemma} \label{thm:gcd lemma}
    For all $a,b,c \in \mathbb{N}$, we have $\gcd(a,bc) = 1 \iff \gcd(a,b) = \gcd(a,c) = 1$.
\end{lemma}
\begin{proof}
    Let $a,b,c \in \mathbb{N}$.
    To prove the forwards implication, suppose that $\gcd(a,bc) \allowbreak = 1$.
    We know that $\gcd(a,b) \mid a$ and $\gcd(a,b) \mid b$, meaning that $\gcd(a,b) \mid bc$.
    Therefore, $\gcd(a,b) = 1$ since $\gcd(a,bc) = 1$.
    By the same reasoning, $\gcd(a,c) = 1$.

    To prove the backwards implication, instead suppose that $\gcd(a,b) = 1$ and $\gcd(a,c) = 1$.
    For the purpose of contradiction, suppose that $\gcd(a,bc) > 1$.
    This means that there exists some prime $p$ such that $p \mid \gcd(a,bc)$.
    This prime must satisfy $p \mid bc$ and thus $p \mid b$ or $p \mid c$ by Euclid's lemma.
    However it must also satisfy $p \mid a$.
    Since $p > 1$, it follows that either $\gcd(a,b) = 1$ or $\gcd(a,c) = 1$ are false.
    This is a contradiction.
    Therefore, $\gcd(a,bc) = 1$.
\end{proof}
\begin{proposition}
    Euler's Totient function is multiplicative.
\end{proposition}
\begin{proof}
    Let $m,n \in \mathbb{N}$ such that $\gcd(m,n) = 1$.
    If $m = 1$ or $n = 1$, then $\varphi(mn) = \varphi(n) = \varphi(m)\varphi(n)$ or $\varphi(mn) = \varphi(m) = \varphi(m)\varphi(n)$ since $\varphi(1) = 1$.
    Otherwise, $m > 1$ and $n > 1$.
    Consider the following table of the natural numbers from 1 to n arranged in $n$ rows of $m$ columns.
    \begin{center}
        \begin{tabular}{|c|c|c|c|c|c|}
            \hline
            1            & 2            & $\cdots$ & $q$          & $\cdots$ & $m$    \\
            \hline
            $m+1$        & $m+2$        & $\cdots$ & $m+q$        & $\cdots$ & $2m$   \\
            \hline
            $2m+1$       & $2m+2$       & $\cdots$ & $2m+q$       & $\cdots$ & $3m$   \\
            \hline
            $\vdots$     & $\vdots$     & $\ddots$ & \vdots       & $\ddots$ & \vdots \\
            \hline
            $m(p-1) + 1$ & $m(p-1) + 2$ & $\cdots$ & $m(p-1) + q$ & $\cdots$ & $mp$   \\
            \hline
            $\vdots$     & \vdots       & $\ddots$ & \vdots       & $\ddots$ & \vdots \\
            \hline
            $m(n-1) + 1$ & $m(n-1) + 2$ & $\cdots$ & $m(n-1) + q$ & $\cdots$ & $mn$   \\
            \hline
        \end{tabular}
    \end{center}
    The $q$-th column is the list of numbers less than or equal to $mn$ that are congruent to $q$ modulo $m$.
    Consider the number $mp + q$ (the entry in the $(p+1)$-th row and $q$-th column).
    Note that for some $d \in \mathbb{N}$, if $d \mid m$ then $d \mid (mp + q) \iff d \mid q$.
    That means that the set of common divisors of $m$ and $mp + q$, and the set of common divisors of $m$ and $q$, are the same.
    Hence, $\gcd(m, mp+q) = \gcd(m, q)$.
    Thus, all the entries in the $q$-th column are coprime to $m$ iff $q$ is coprime to $m$.
    There are $\varphi(m)$ such columns.

    Consider the map $p \mapsto m(p-1) + q \pmod n$ for some fixed $q$.
    By \textbf{Bez\'out's identity}, there exists $x,y \in \mathbb{Z}$ such that $mx + ny = \gcd(m,n) = 1$.
    Thus, $mx = 1 - ny \equiv 1 \pmod n$, so $m$ has a multiplicative inverse modulo $n$.
    That means that the map is injective, and since the domain and codomain are of the same size, it is surjective.
    Therefore, the entries of any given column cover all the residue classes modulo $n$ exactly once, so each column has $\varphi(n)$ entries coprime to $n$.

    In total, there are thus $\varphi(m)\varphi(n)$ numbers coprime to both $m$ and $n$
    By Lemma~\ref{thm:gcd lemma}, there are $\varphi(mn)$ numbers that are coprime to both $m$ and $n$.
    Therefore, $\varphi(mn) = \varphi(m)\varphi(n)$.
\end{proof}
\begin{proof}[Alternative Proof using Group Theory]
    Let $m,n \in \mathbb{N}$ such that $\gcd(m,n) = 1$.
    Note that, in general, $\mathbb{Z}/n\mathbb{Z} \cong C_n$.
    Therefore, $\mathbb{Z}/(mn)\mathbb{Z} \cong \mathbb{Z}/m\mathbb{Z} \times \mathbb{Z}/n\mathbb{Z}$ as $C_m \times C_n \cong C_{mn} \iff \gcd(m,n) = 1$.
    This is essentially the \textbf{Chinese remainder theorem} for two coprime numbers.

    If $c_1 \in C_1$ is a generator of $C_1$, and $c_2 \in C_2$ is a generator of $C_2$, then $(c_1, c_2)$ is a generator of $C_1 \times C_2$.
    Now, $\varphi(n)$ is the number of elements in $\mathbb{Z}/n\mathbb{Z}$ that are generators.
    Therefore, $\varphi(mn) = \varphi(m)\varphi(n)$ since the chocie of a generator of $\mathbb{Z}/(mn)\mathbb{Z}$ corresponds to the choice of a generator of $\mathbb{Z}/m\mathbb{Z}$ and a generator of $\mathbb{Z}/n\mathbb{Z}$.
\end{proof}
\end{document}